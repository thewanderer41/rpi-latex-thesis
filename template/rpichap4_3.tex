%%%%%%%%%%%%%%%%%%%%%%%%%%%%%%%%%%%%%%%%%%%%%%%%%%%%%%%%%%%%%%%%%%%
%                                                                 %
%                            CHAPTER SIX                          %
%                                                                 %
%%%%%%%%%%%%%%%%%%%%%%%%%%%%%%%%%%%%%%%%%%%%%%%%%%%%%%%%%%%%%%%%%%%

\chapter{CHANGE DISTANCE} \label{ch:distance}

\section{GCMD}

\subsection{GCMD Versioning Graph}

The Global Change Master Directory establishes the context that each \textbf{manifestation} of their keyword list are related versions.
Since the unique identifier for each keyword remains the same across versions, they can be used to align a mapping across versions.
\textbf{Additions} and \textbf{invalidations} are detected by checking an identifier's presence within both versions.
A \textbf{modification} occurs when a keyword's \textit{skos:Broader} property differs between adjacent versions.
A difference indicates that the word has been moved to a different place within the taxonomy since identifiers do not change across versions and a keyword only has one parent concept.
Changes over consecutive versions can be collected into a single graph using the method in Section \ref{sec:multiver} to chain together versioning graphs.
A change log was generated for each pair of consecutive versions in GCMD Keywords and embedded with JSON-LD.
Versioning graphs for each adjacent version was created by extracting JSON-LD from the corresponding change log, and entering the triples into a Fuseki triple store.

\subsection{Connecting Change Counts to Identifiers}

The \textbf{add}, \textbf{invalidate}, and \textbf{modify} counts for each transition are presented in Figure \ref{GCMDC1}.
The query used to extract the counts is found in Listing \ref{gcmd_list}.
Notice the sharp spike in adds and invalidates when transitioning from version 8.4.1 to 8.5.
The version identifiers indicate that at most a minor or technical change has occurred, but the counts of \textbf{addition} and \textbf{invalidation} changes in this transition is more than triple the counts in either of the previous \textbf{major} transitions.
Not only should a small transition not produce changes of this quantity, but the data set's size is on the order magnitude of the recorded \textbf{invalidates}.
In addition, no \textbf{modifications} are revealed, and even the root node "Science Keywords" has been invalidated.
Further investigation of the root word reveals that the name space for the keywords has changed from HTTP to HTTPS.
To provide context, NASA mandated a transition to secure protocols, and the group changed the name space to ensure the URIs remained resolvable.
Since the identifiers are unique, the new name space means they no longer refer to the same object after the protocol change.
Because the keyword identifiers no longer match, the mapping approach results in the total invalidation of keywords from 8.4.1 and the addition of keywords from 8.5.
The dot decimal identifier for the transition from version 8.4.1 to 8.5 does not match the number of changes in the versioning graph.

\begin{figure}[b]
	\centering
	\includegraphics[scale=0.91]{figures/GCMDChart1.png}
	\caption{Add, Invalidate, and Modify counts in Version 8.5.  The counts show change magnitudes and indicate that major and minor changes differ by orders of magnitude.}
	\label{GCMDC1}
\end{figure}

%\hfill \break
\begin{lstlisting}[language=SPARQL, caption=This query compiles the counts for each subclass of Change in a GCMD versioning graph,label=gcmd_list]
PREFIX vo:<http://orion.tw.rpi.edu/~blee/VersionOntology.owl>
PREFIX rdfs:<http://www.w3.org/2000/01/rdf-schema#>

SELECT ?p (COUNT (DISTINCT ?s) as ?count)
{
?s a ?p .
?p rdfs:subClassOf vo:Change .
} GROUP BY ?p
\end{lstlisting}

Changing the mapping method to account for the new namespace provides a pathway to compare the perceived change by the producer as evidenced by the version identifier with the amount of change in the versioning graph.
To do this, the mapping treats identifiers with HTTP and HTTPS the same. 
Differences in change magnitudes become much clearer after controlling for the altered name space in Figure \ref{GCMDC2}.
All revisions are dominated by \textbf{additions}, but major version changes have counts around 300 to 500 while minor revisions are an order of magnitude smaller.
The transition from version 8.4.1 to 8.5 also seems to follow this trend.
The \textbf{additions} in ``8.4 to 8.4.1" in Figure \ref{GCMDC2} numbers almost a hundred, providing evidence that the trend of decreasing order of magnitudes may now continue as the granularity of the version identifier increases.

\begin{figure}%[b]
	\centering
	\includegraphics[scale=1]{figures/GCMDChart2.png}
	\caption{Add, Invalidate, and Modify counts ignoring the namespace changes in Version 8.5.  The counts show change magnitudes appropriate for the identifier.}
	\label{GCMDC2}
\end{figure}

\section{MBVL}


\subsection{Variant Versioning Graph}

The experiment conducts activity over two phases in this procedure.
The first phase takes sequences from the original known population and feeds the sequences though a particular algorithm/taxonomy combination to produce a candidate classification.
Since the classifications for the known population sequences is unavailable, there is not sufficient context to perform a valid comparison with the candidate classifications.
The second phase compares the performances of each candidate classification of a algorithm/taxonomy pair.
The use of \textbf{add}, \textbf{invalidate}, and \textbf{modify} varies slightly in this application since all the results use the same sequences.
A versioning graph utilizing just the sequence identifiers would only result in \textbf{modify} changes when taxonomic ranks differ since the sequence identifier exists in both data sets.
The mapping instead uses the sequence identifiers to align comparisons and then the taxonomic rank classification to determine the kind of change.
If the right-hand result specifies more taxonomic ranks, the relationship is an \textbf{addition}.
If the left-hand result is more specific, then the relationship is classified as an \textbf{invalidation}.
If both results have the same precision but the name differs, then the link is a \textbf{modification}.
Otherwise, no change is detected.

Figure \ref{mbvl_chart} shows the changes detected when varying either the taxonomy or the classification algorithm.
\begin{figure}
	\centering
	\includegraphics[scale=0.75]{figures/mbvl_chart.png}
	\caption{Compiled counts of \textbf{adds}, \textbf{invalidates}, and \textbf{modifies} grouped by taxonomic rank across algorithm and taxonomy combinations.}
	\label{mbvl_chart}
\end{figure}
No comparison was conducted with different taxonomy and classifier since that would introduce too many sources of variability to differing results in a classification.
Each bar indicates the total number of differences between sequences for a specific kind of change.
The bars are further broken down by the taxonomic rank at which the difference occurred.
For example, in ``Silva vs RDP, Gast", a notable number of classifications differed at the species rank.
The graph also indicates that using the RDP taxonomy often produces more precise classifications since both ``Silva vs RDP, Spingo" and ``Silva vs RDP, Gast" feature a larger number of \textbf{additions} than any other change.
The classifier comparisons feature a high number of \textbf{invalidations}; however, ``RDP, Spingo vs Gast" also displays a higher number of \textbf{modifications} than \textbf{invalidations}.

