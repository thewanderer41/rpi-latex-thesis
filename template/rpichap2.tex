%%%%%%%%%%%%%%%%%%%%%%%%%%%%%%%%%%%%%%%%%%%%%%%%%%%%%%%%%%%%%%%%%%%
%                                                                 %
%                            CHAPTER TWO                          %
%                                                                 %
%%%%%%%%%%%%%%%%%%%%%%%%%%%%%%%%%%%%%%%%%%%%%%%%%%%%%%%%%%%%%%%%%%%

\chapter{LITERATURE REVIEW}\label{ch:prevwork}
%\resetfootnote %this command starts footnote numbering with 1 again.

The data versioning landscape produces a variety of different approaches and standards towards change capture.
However, massive centralized data stores have become more prevalent as data distribution methods advance  \cite{Vassiliadis1999}.
Collection into larger unified repositories will likely require a multi-tiered approach to synchronize the varied practices  \cite{Baker2009}.
Baker notes that differences depend on the sociotechnical distance of a repository from the data's origin \cite{Baker2009}.
Local stores closer to the collection site better understand data capture conditions, but must also adapt to changing environments.
Science agencies and organizations are only beginning to formally codify and standardize methods to capture and publish lineage information \cite{MatthewS.Mayernik201312-039}.
This seems to be a recurring cycle of attention with rapidly developing technologies such as the grid or parallel computing \cite{Kovse2003VGridAVS}.
The CERN grid for the Compact Muon Solenoid experiment carefully developed necessary processes which allow references by multiple users to the same file without copying that file across the grid \cite{Holtman:687353}.

The model defined in Chapter \ref{ch:model} provides a basis for understanding the formal underlying properties which will allow consistent versioning practices.
Previous work in provenance, provenance distance, and mapping provide inspiration for the model's form.
A majority of the linked data approaches to version capture can be found in provenance models.
Additionally, researchers have attempted to address the amount of change between version identifiers using a measure called provenance distance.
This distance measure lacks sufficient resolution to provide detailed quantities to answer many versioning questions.
The model also draws inspiration from existing change mapping methods found in version control managers to reduce space consumption when working with large data sets.
These mapping methods provide a familiar and regularized basis to both compute and present changes.

\section{Data Versioning Operations}

Versioning systems come in a variety of different forms as a result of the assorted applications they must manage.
Experimental biology data require tracking as they pass through a scientific workflow \cite{Tagger2005}.
Libraries curate multiple editions of the same work, sometimes with significant revisions \cite{Wiil:2000:RDH:338407.338517}.
What such activities demonstrate is the wide range of settings and expectations under which versioning systems operate.
While this work cannot hope to explore all these applications, every system studied shares three common functions: addition, deletion, and modification.
These three operations plausibly form the core set of relationships when versioning.

Literature surveys often expect versioning systems to interact with data uniformly because they are asked to perform the same functions \cite{Tagger2005}.
However, different data sets may utilize each of the three core operations at different rates \cite{rohtua}.
This helps to characterize the data set in ways such as a growing set with many additions, a stable collection featuring occasional corrections, or a wildly volatile data set consisting of often deleted and replaced data files.
Understanding these would give insight into the maturity and health of a data set, but most major provenance ontologies do not heavily feature these relationships.
Versioning information, of course, does not give insight into the origins of an object, and therefore, would not be expected to play a major role in provenance documentation.

While data addition and modification remain fairly uncontroversial, there is a mild division between practical and theoretical approaches to data deletion \cite{Flouris04clotho:transparent}.
A removed object provides evidence of an erroneous activity's results or intermediary steps leading to a final product.
As a result, version management should maintain and track invalidated data.
The software versioning manager GIT uses a method of compressing older data to conserve space without deleting the data \cite{Chacon:2009:PG:1618548}.
Available storage space places pragmatic constraints on the number of projects which can adopt snapshotting practices.
In applications which cannot recover erroneous data nor use it as documentation artifacts, like corrupted surveillance images.
Some high energy physics experiments cannot re-collect observational data due to cost, and as a result, they cannot replace or re-process poor quality data \cite{Cavanaugh2002}.
While the distinction between `deletion' and `invalidations' remains largely semantic, the terms' use in this document reflects an understanding of the different constraints and requirements placed on versioning systems.
As a result, invalidation is adopted as a broad, general term to also encompass data deletions.

A handful of other operations exist among version managers, but they do not prove ubiquitous across most applications.
Software versioning tools like RCS commonly feature branching and merging functions to create a versioning line separate from the stable master branch \cite{tichy1985rcs}.
This mostly provides an organizational role in development by allowing developers to experiment without contaminating a stable software release.
Figure \ref{GITTree} models this, showing versions C3 and C5 in branch iss53 before being merged back into the production line as C6.
It allows for more orderly management of versions, but does not conduct versioning itself.
Other activities provide functional operations such as locking and unlocking files from edits to prevent race conditions in branch mergers.
It does not introduce any new relationships but allows the tool to operate more smoothly.
Clotho, a versioning application managing versions at the block level, coordinates constrained spaces using intermittent compact and un-compact methods when retrieving or storing old objects \cite{Flouris04clotho:transparent}.
Likewise, many version control tools include functions to display the versioning tree, but this is also an ease-of-use function \cite{Dijkstra1994}.

\begin{figure}
	\centering
	\includegraphics[scale=0.75]{figures/GITCommitTree.png}
	\caption[Example of a commit history with branching stored in GIT.]{Example of a commit history with branching stored in GIT.  Figure 3.17 from \cite{Chacon:2009:PG:1618548}}
	\label{GITTree}
\end{figure}

\subsection{Types of Change}

Barkstrom uses the ability to scientifically distinguish between two data sets as a criteria for major divisions among groupings \cite{Barkstrom2003}.
At lower levels, he notes that science teams can no longer discern scientific differences between data sets.
Instead, they observe that changes to format and structure contribute significant alterations without changing any values withing the data.
As a result, these technical changes form a second boundary to meaningfully separate minor version groupings.
Finally, the explicit values may need occasional revisions to correct lexical errors such as spelling or formatting.
These terms were chosen carefully as they reflect the three value dot-decimal identifier system.
However, data producers will often use qualitative measures to determine the type of change occurring between versions.
Versioning system users wish to achieve insight into the type of change that occurs between versions, and quantitative analysis on versioning operations will provide quantifiable evidence in characterizing the impact of a change.

The exact category that a particular change falls into can be controversial.
The decision to provide concentration units from parts per million to milligrams per milliliter poses a Technical change for a data producer.
However, for a data consumer, the alteration may be viewed as a Scientific change as it invalidates the methods they had previously used.
This conflict in view illustrates the data consumer-producer dynamic.
In general, data producers control the versioning methods, but data consumers determine the classification of a data change.
Producers tend to use versioning systems to ensure data quality of service through audits and recovery tools \cite{Cavanaugh2002}.
Meanwhile, a consumer will analyze the historical changes and determine the impact this may have on their data use.
As a result, this means that data versioning systems must communicate a dynamic view of the changes in a system contextualized by the user of that data.

Version managers often disagree at the point many technical changes sufficiently modifies a data set that it comprises a scientific change.
As determining changes in science requires expert understanding over a domain, different measures should be explored to address the distinction.
This document's approach studies add, invalidate, and modify counts to quantify the impact of changes and how they relate to the producer's observations.
As previously mentioned, different systems utilize the operations at varying rates so absolute cutoffs will be unlikely in comparison to relative results.

\section{Provenance Distance}

With increasing complexity, data workflows have developed in such a way that even subtle changes have serious implications for other parts of the workflow \cite{TILMES2011548}.
This observation makes change impact difficult to measure, but one insight begins with provenance's role in workflows.
Provenance can give great insight into a data object's future performance such as the  ability to predict disk usage based on the lineage of a data object \cite{dai2014provenance}.
Efforts have also been made to summarize provenance representations to improve consumption \cite{Ainy:2015:ASD:2806416.2806429}.
Changes to the process creating an object signals the development of a new version.
Therefore, studying the magnitude of this deviation should give some idea into the resulting object's impact.
This idea, known as provenance distance, seeks to determine the impact of changes in provenance on new data versions through measuring graph edit distances.

\begin{figure}
	\centering
	\begin{adjustbox}{addcode={\begin{minipage}{\width}}{
					\caption[Provenance graph of a Level 3 data product, showing the inter-relations between different data products in generating the final product.]{Provenance graph of a Level 3 data product, showing the inter-relations between different data products in generating the final product.  Figure 2 from \cite{TILMES2011548}}\end{minipage}},rotate=90,center}
		\includegraphics[scale=0.5]{figures/OzoneProvGraph.png}
	\end{adjustbox}
	\label{ProvGraph}
\end{figure}

The first ingredient necessary to calculate provenance distance is a linked data graph capturing the sequence of events leading to the old and new objects' creation, like the one shown in Figure \ref{ProvGraph}.
It shows the multiple lower level products involved in creating a Level 3 ozone indicator.
This can be accomplished through the use of previously mentioned provenance models, but these graphs are not widely available.
Using PROV to represent provenance data in a semantic model produces an acyclic directed graph with labeled nodes.
As a result, the provenance distance problem reduces to similarity measurement.
When calculating this measure, algorithms determine how far two graphs are from being isomorphic \cite{Cao2013}.
Node labeling simplifies this process by providing nodes which must match together, and greatly reduces the complexity from computing generalized graphs.
Graph Edit Distance, counting the edits necessary to transform one graph into another, provides a quantitative measure to associate with this process  \cite{Gao2010}.
Some variations count edge changes \cite{Goddard:1996:DGU:246962.246972}.

In Figure \ref{GraphEdit}, the left graph transforms through a move of edge 1 and a rotation of edge 4, resulting in an edit distance of two.
Such changes in a provenance graph would demonstrate an alteration in dependencies between objects used to generate a final notable product.
This kind of analysis resembles comparison measures employed in determining semantic similarity \cite{Hliaoutakis06informationretrieval}.
However, isolating changes responsible for differences in provenance can become difficult in complex environments as Tilmes observes in 2011, 
\begin{quotation}
	Consider the relatively common case of the calibration table, which is an input to the L1B process, changing. Even though the version of the L2 or L3 software hasn't changed, the data files in the whole process have been affected by the change in the calibration.
\end{quotation} \cite{TILMES2011548}.
L-number is shorthand for the level system featured in Figure \ref{NASALevels}.
While provenance distance may be straight-forward to calculate, the indicator hides many insights into an object's behavior.

\begin{figure}
	\centering
	\includegraphics[scale=0.40]{figures/GraphEdit.png}
	\caption[The labeled graph on the left transforms into the right graph under two edge edits.]{The labeled graph on the left transforms into the right graph under two edge edits. Figure 2 from \cite{Goddard:1996:DGU:246962.246972}}
	\label{GraphEdit}
\end{figure}

Methods to provide quality of service boundaries leveraging provenance already exist which compare workflows based on performance criteria \cite{2015:CAA:2778374.2778504}.
However, these procedures focus primarily on quick retrieval and efficient storage instead of capitalizing on the latent information accessed by reasoning across data set versions \cite{tan2004research}.
The distance measures previously mentioned rely solely on provenance graphs to compute results, but this is obviously insufficient.
When considering the provenance of a data object, methods only consider the activities and entities that took an active role in the production of it.
A new version of an object has a familial relationship with its previous versions, but in most cases, they do not take an active role in its generation.
Without detailed change information, determining the difference between two data objects in a metric beyond broad strokes becomes difficult, if not impossible.

As per our definition of `version', objects must have common provenance, and the more similar they are, the more meaningful the results from versioning methods.
Provenance distance provides a means of determining how reliable versioning results are given a greater adoption of provenance graphs.
Measuring a change's impact with accuracy comparable to a change log requires a more detailed understanding and description than provenance can provide  \cite{Bose:2005:LRS:1057977.1057978}.
Sufficiently precise versioning measurements cannot be provided by provenance distance, but it could indicate the confidence of versioning results, which is out of scope for this project.

\section{Mapping}

Data managers primarily use one of two methods to store data versions: snapshots and deltas.
The snapshot method makes periodic copies of the data's state at a point in time.
While storing and retrieving these snapshots can be very quick, they require significant amounts of space to maintain.
The software manager GIT employs this method and Figure \ref{GITFile} demonstrates an example storage space for multiple versions \cite{Chacon:2009:PG:1618548}.
The squares with dotted outlines indicate unmodified files, which the system stores as pointers instead of full objects.
In addition, GIT compresses and separately stores very old versions which are unlikely to be accessed.
This versioning style may not be ideal for larger or often modified data sets as the size requirements will quickly grow unmanageable.
However, for many library or catalog environments, they cannot predict the target volume a user desires and must prioritize availability \cite{Payette2002} \cite{Barkstrom_digitallibrary}.
Some methods like the inverted file index have been developed to balance space and retrieval performance on web documents, especially since wikis and news feeds have grown in deployment \cite{Berberich:2007:TMT:1277741.1277831}.
Searches over these text media may require execution on older archived web pages.

\begin{figure}
	\centering
	\includegraphics[scale=0.50]{figures/GITFiles.png}
	\caption[GIT stores changes in the repository as snapshots of individual files.]{GIT stores changes in the repository as snapshots of individual files. Figure 1.5 from \cite{Chacon:2009:PG:1618548}}
	\label{GITFile}
\end{figure}

The delta method entails calculating and storing only the differences between one version and the next.
Back delta variations store a snapshot of the most recent version and compute deltas towards older releases.
The forward delta variation stores the oldest data's snapshot and has deltas going forwards.
This method uses the minimum amount of space but trades it in for computation time to recreate any given version.
Particularly long running versioning systems occasionally save an intermittent snapshot to cut down on this processing time.
The setup proves ideal for data sets which prioritize service to their most recent versions \cite{Stuckenholz:2005:CEV:1039174.1039197}.
Because change documentation captures information between version objects, they most resemble differences calculated by the delta method.

Properly detecting changes in a system's files allows file managers to correctly group them into versions as seen in research conducted by the Atmospheric Radiation Measurements (ARM) group \cite{6906868}.
Difference or diff applications must first properly map data between objects and align them for comparison.
Many text-based data sets rely on well-established algorithms to perform this alignment  \cite{Chien:2000:VMX:646544.696357} \cite{Hartung201315}.
Sequential scientific data largely avoids this problem since developers already know the files or objects they replaced.
However, users do not have this advantage and system managers are starting to recognize the difference in versioning usage patterns between users and producers \cite{Branco2008}.
Mayernik, et al., probably gives the best description saying, "Prospective records document a process that must be followed to generate a given class of products whereas retrospective records document a process that has already been executed" \cite{MatthewS.Mayernik201312-039}.
While producers take a retrospective approach to version usage, consumers of new versions must take a prospective view, adapting to new changes.
This indicates that the orientation of versioning information reflects the imagined customer of that data.

Current linked data methods lack the fidelity to capture change information.
PROV and OPM express version changes as a single link, but capturing individual changes provides better detail, allowing more accurate distance measures.
Measuring change this way better communicates the change existing within a versioning system than courser methods leveraging only provenance graphs.
Implementing the version capture requires the procedures used to perform difference calculations and going a step further by creating linked data statements.
The model in Chapter \ref{ch:model} defines concepts to capture individual changes based on common operations and compares them with current methods of capture.

%%% Local Variables:
%%% mode: latex
%%% TeX-master: t
%%% End:
