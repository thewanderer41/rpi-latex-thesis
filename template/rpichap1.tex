%%%%%%%%%%%%%%%%%%%%%%%%%%%%%%%%%%%%%%%%%%%%%%%%%%%%%%%%%%%%%%%%%%%
%                                                                 %
%                            CHAPTER ONE                          %
%                                                                 %
%%%%%%%%%%%%%%%%%%%%%%%%%%%%%%%%%%%%%%%%%%%%%%%%%%%%%%%%%%%%%%%%%%%

\chapter{INTRODUCTION}

Software development followed a stiff production cycle prior to the early 2000s.  However, as technology developed, software development required a system to adapt and change to evolving conditions leading to the Agile Manifesto.  Likewise, data collected by researchers grew at an astounding rate with new technology.  NASA's Atmospheric Science Data Center reported a growth from hosting around five million files to twenty million files between 2001 and 2004.  Many datasets have required reprocessings of their data, either to improve data quality or to correct for errors [CALIPSO, ARM].   

Dataset versioning tracks and documents the changes which occur in datasets.  Version numbers are commonly seen in relation to new releases of software such as MATLAB or R, but current version naming schemes borrow from software contexts.  While it may have been possible to manually manage the changes to data when datasets were relatively small, research centers must now supervise on the order of tens of millions of files with changes being made at rates of thousands of processing jobs per day.

Versioning information is widespread across devices and software in the modern world.  From the numbering of the latest smart phone to the patch number of the newest release of MATLAB, scientists must deal with a range of labels and formats when performing their research.  Science data, likewise, also has a tendency to change.  Datasets are subjected to data audits and error corrections regularly to maintain a level of data quality.

Agencies and research groups have collected new data at an incredible rate.  The amount of data housed by NASA quadrupled from 2001 to 2004 \cite{barkstromLibrary} and high energy physics labs can generate on the order of 4000 new datasets every day \cite{ATLAS}.

\section{Provenance}
In a number of papers, authors describe models or systems that track changes in the workflow and how modifications to the flow would then generate new versions of datasets.  The information that details the activities and agents involved in generating a data entity is known as provenance.  Barkstrom identifies times when new versions should be generated by locating changes in provenance (specifically calibrations, scripts, and input data) at different levels of data processing in NASA remote sensing workflows.  Software revision management tools such as Git and SVN also keep track of provenance information when logging new commits to a project.  Since it plays a significant role in triggering new version generations, provenance is often conflated with versioning by scientists.  However, it does not sufficiently characterize a versioning event.

Provenance is the data used to describe the origin of an object (Merrium Webster).  In the field of semantic technologies, the W3C recommendation, PROV provides a data model to encode provenance information so that the lineage of data products can be traced.  However, PROV expresses relationships between versions with the wasRevisionOf or alternateOf property.  The properties do not allow for more elaboration as to what changes were made to transform version one into its alternate, nor should it.  Explicitly itemizing the changes is unnecessary in order to communicate the provenance relationship between the two versions.  That there exists an association is sufficient.  This gap illustrates why versioning information is necessary because itemization allows data consumers to determine the significance of changing to version two.

\section{Types of Change}

One of the challenges facing data versioning is that changes within a version often mean different things to different people.  From a data producers standpoint, changing the units within a dataset does not have significant impact to the science they are trying to communicate, but it may have catastrophic impacts on a data consumer who is expecting the dataset to be formatted in specific units.

\section{Changelogs}

\section{RDFa}

\section{Why Semantic Technologies?}

This is a sentence to take up space \cite{thisbook}.
This is a sentence to take up space and look like text.
This is a sentence to take up space and look like text.

Please refer to Figure~\ref{myfig}.  % Note \label command below

\begin{figure}
\centering
\vspace{2.0in} % normally the command here would be \includegraphics
\caption{This is the Caption for Figure 1 make it long to illustrate
how it looks when wrapped around to the next line}
\label{myfig}  % the \label command comes AFTER the caption
\end{figure}

This is a sentence to take up space and look like text.
This is a sentence to take up space and look like text.
This is a sentence to take up space and look like text.

\begin{table}
\caption[This is the Caption for Table 1]
            {This is the Caption for Table 1\cite{thisbook}}
%  Note entry in [] for list of tables; you don't want the citation in the LOT.
\begin{center}
\begin{tabular}{lll}
Here's       & an          & example  \\
of           & a           & table    \\
floated      & with        & the      \\
\verb+table+ & environment & command.
\end{tabular}
\end{center}
\end{table}

This is a sentence to take up space and look like text.
This is a sentence to take up space and look like text.

\section{This is a Section Heading}

This is a sentence to take up space and look like text.
This is a sentence to take up space and look like text.
This is a sentence to take up space and look like text.

\subsection{This is a Subsection Heading}

This is a sentence to take up space and look like text.
This is a sentence to take up space \cite{anotherbook}.
This is a sentence to take up space and look like text.

\subsubsection{This is a Subsubsection Heading}
This is a sentence to take up space and look like text.
This is a sentence to take up space and look like text.
Text before the footnote.\footnote{Here's the text of the footnote.}
Text after the footnote.
This is a sentence to take up space and look like text.

%%% Local Variables:
%%% mode: latex
%%% TeX-master: t
%%% End:
