%%%%%%%%%%%%%%%%%%%%%%%%%%%%%%%%%%%%%%%%%%%%%%%%%%%%%%%%%%%%%%%%%%%
%                                                                 %
%                            CHAPTER ONE                          %
%                                                                 %
%%%%%%%%%%%%%%%%%%%%%%%%%%%%%%%%%%%%%%%%%%%%%%%%%%%%%%%%%%%%%%%%%%%

\chapter{INTRODUCTION}

Software development followed a stiff production cycle prior to the early 2000s.  However, as technology developed, software development required a system to adapt and change to evolving conditions leading to the Agile Manifesto.  Likewise, data collected by researchers grew at an astounding rate with new technology.  NASA's Atmospheric Science Data Center reported a growth from hosting around five million files to twenty million files between 2001 and 2004\footnote{Agencies and research groups have collected new data at an incredible rate.  The amount of data housed by NASA quadrupled from 2001 to 2004 \cite{barkstromLibrary} and high energy physics labs can generate on the order of 4000 new datasets every day \cite{ATLAS}.}.  Many datasets have required reprocessings of their data, either to improve data quality or to correct for errors [CALIPSO, ARM].  Indeed, version control systems can provide contexts for modifications when combined with error detection systems, providing a more complete picture of the system's behavior [Reference 9].  (Dynamic dataset generation?)

Dataset versioning tracks and documents the changes which occur in datasets.  There exists a tendancy to model dataset change using methods from software versioning.  Dot-decimal style version numbers are commonly seen in relation to new releases of software such as MATLAB or R, and current version naming schemes in data often use dot-decimal notation.  However, this convention results in semantically poor version names as the decimal numbers generally function solely as counts.  The Ubuntu versioning scheme appears to follow a major-minor numbering pattern, but the major and minor numbers correspond to the year and month of the release, respectively.  As a result, the version name holds no significance as the extent of difference between two versions.  Additionally, this method breaks down since it is difficult to predict the impact changes in data will have on data consumers.  Depending on how the data is processed or what subsets are used, dataset changes that the producers perceives as minor changes may have significant repercussions for the consumer.  This demonstrates the structural difference between software and datasets in that a revision in data only constitutes an adjustment to the relevant subsets.  An alteration to a software file constitutes a change to the project as a whole.  As a result, datasets provide a context which does not completely translate from software project versioning.



\section{Types of Change}

Change is the fundamental characteristic in the study of versioning.  Continuing the comparison with software, the measure of change in applications is primarily functional.  Data, on the other hand, has not only function but also structure.  While swapping the ordering of columns in a spreadsheet may not effect the usability of the data, it could cause issues with codes that use the data.  It is, therefore, clear that many different forms of change exist with differing degrees of intensity.  In this document, change is viewed as having three general categories: scientific, technical, and lexical.  The most impactful change, scientific, denotes changes that have modified the fundamental science used to generate the dataset.  Changes in algorithms or sampling methods generally fall under scientific modifications as they effect the base assumptions and possible error within the dataset.  Technical impacts do not change the underlying science of the data, but impose a large enough change as to warrant notice.  Structure alteration and unit conversions count as technical changes since the dataset now needs to be consumed differently but remains valid for use.  

\section{Provenance and Provenance Distance}

In a number of papers, authors describe models or systems that track changes in the workflow and track how modifications to the flow would then generate new versions of datasets.  The information that details the activities and agents involved in generating a data entity is known as provenance.  Barkstrom identifies times when new versions should be generated by locating changes in provenance (specifically calibrations, scripts, and input data) at different levels of data processing in NASA remote sensing workflows.  Software revision management tools such as Git and SVN also keep track of provenance information when logging new commits to a project.  Since it plays a significant role in triggering new version generations, provenance is often conflated with versioning by scientists.  However, it does not sufficiently characterize a versioning event.

In the field of semantic technologies, the W3C recommendation, PROV provides a data model to encode provenance information so that the lineage of data products can be traced.  However, PROV expresses relationships between versions with the wasRevisionOf or alternateOf property.  The properties do not allow for more elaboration as to what changes were made to transform version one into its alternate, nor should it.  The priority of provenance tracking is understanding where and how a data object was generated.  Explicitly itemizing the changes is unnecessary in order to communicate the provenance relationship between the two versions.  That there exists an association is sufficient within the scope of lineage tracking.  This gap illustrates why versioning information is necessary because itemization allows data consumers to determine the significance of changing to a new version.

\section{Changelogs}

\section{RDFa}

\section{Why Semantic Technologies?}

This is a sentence to take up space \cite{thisbook}.
This is a sentence to take up space and look like text.
This is a sentence to take up space and look like text.

Please refer to Figure~\ref{myfig}.  % Note \label command below

\begin{figure}
\centering
\vspace{2.0in} % normally the command here would be \includegraphics
\caption{This is the Caption for Figure 1 make it long to illustrate
how it looks when wrapped around to the next line}
\label{myfig}  % the \label command comes AFTER the caption
\end{figure}

This is a sentence to take up space and look like text.
This is a sentence to take up space and look like text.
This is a sentence to take up space and look like text.

\begin{table}
\caption[This is the Caption for Table 1]
            {This is the Caption for Table 1\cite{thisbook}}
%  Note entry in [] for list of tables; you don't want the citation in the LOT.
\begin{center}
\begin{tabular}{lll}
Here's       & an          & example  \\
of           & a           & table    \\
floated      & with        & the      \\
\verb+table+ & environment & command.
\end{tabular}
\end{center}
\end{table}

This is a sentence to take up space and look like text.
This is a sentence to take up space and look like text.

\section{This is a Section Heading}

This is a sentence to take up space and look like text.
This is a sentence to take up space and look like text.
This is a sentence to take up space and look like text.

\subsection{This is a Subsection Heading}

This is a sentence to take up space and look like text.
This is a sentence to take up space \cite{anotherbook}.
This is a sentence to take up space and look like text.

\subsubsection{This is a Subsubsection Heading}
This is a sentence to take up space and look like text.
This is a sentence to take up space and look like text.
Text before the footnote.\footnote{Here's the text of the footnote.}
Text after the footnote.
This is a sentence to take up space and look like text.

%%% Local Variables:
%%% mode: latex
%%% TeX-master: t
%%% End:
