%%%%%%%%%%%%%%%%%%%%%%%%%%%%%%%%%%%%%%%%%%%%%%%%%%%%%%%%%%%%%%%%%%%
%                                                                 %
%                            CHAPTER ONE                          %
%                                                                 %
%%%%%%%%%%%%%%%%%%%%%%%%%%%%%%%%%%%%%%%%%%%%%%%%%%%%%%%%%%%%%%%%%%%

\chapter{INTRODUCTION}

Dataset versioning is the process by which changes to data are tracked and documented.  Version numbers are commonly seen in relation to new releases of software such as MATLAB or R, but current version naming schemes borrow from software contexts.  While it may have been possible to manually manage the changes to data when datasets were relatively small, research centers must now supervise on the order of tens of millions of files with changes being made at rates of thousands of processing jobs per day.

Versioning information is widespread across devices and software in the modern world.  From the numbering of the latest smart phone to the patch number of the newest release of MATLAB, scientists must deal with a range of labels and formats when performing their research.  Science data, likewise, also has a tendancy to change.  Datasets are subjected to data audits and error corrections regularly to maintain a level of data quality.

Agencies and research groups have collected new data at an incredible rate.  The amount of data housed by NASA quadrupled from 2001 to 2004 \cite{barkstromLibrary} and high energy physics labs can generate on the order of 4000 new datasets every day \cite{ATLAS}.

Provenance is the data used to describe the origin of an object (Merrium Webster).  While data provenance and data versioning are related, the priorities of one should not be confused with the other.  Understanding the origins of a dataset and the activities leading to its creation may inform why two data objects are different, the changes made do not take priority.  Many different views of versioning has appeared in the literature.  When Barkstrom and various other experts refer to data versioning, they are actually referring to data provenance.


This is a sentence to take up space \cite{thisbook}.
This is a sentence to take up space and look like text.
This is a sentence to take up space and look like text.

Please refer to Figure~\ref{myfig}.  % Note \label command below

\begin{figure}
\centering
\vspace{2.0in} % normally the command here would be \includegraphics
\caption{This is the Caption for Figure 1 make it long to illustrate
how it looks when wrapped around to the next line}
\label{myfig}  % the \label command comes AFTER the caption
\end{figure}

This is a sentence to take up space and look like text.
This is a sentence to take up space and look like text.
This is a sentence to take up space and look like text.

\begin{table}
\caption[This is the Caption for Table 1]
            {This is the Caption for Table 1\cite{thisbook}}
%  Note entry in [] for list of tables; you don't want the citation in the LOT.
\begin{center}
\begin{tabular}{lll}
Here's       & an          & example  \\
of           & a           & table    \\
floated      & with        & the      \\
\verb+table+ & environment & command.
\end{tabular}
\end{center}
\end{table}

This is a sentence to take up space and look like text.
This is a sentence to take up space and look like text.

\section{This is a Section Heading}

This is a sentence to take up space and look like text.
This is a sentence to take up space and look like text.
This is a sentence to take up space and look like text.

\subsection{This is a Subsection Heading}

This is a sentence to take up space and look like text.
This is a sentence to take up space \cite{anotherbook}.
This is a sentence to take up space and look like text.

\subsubsection{This is a Subsubsection Heading}
This is a sentence to take up space and look like text.
This is a sentence to take up space and look like text.
Text before the footnote.\footnote{Here's the text of the footnote.}
Text after the footnote.
This is a sentence to take up space and look like text.

%%% Local Variables:
%%% mode: latex
%%% TeX-master: t
%%% End:
