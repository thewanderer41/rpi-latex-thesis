
\chapter{DATASETS UTILIZED}

\section{Noble Gas Dataset}

The ``Global Database on \textsuperscript{3}He/\textsuperscript{4}He in on-shore free-circulated subsurface fluids" is a tumultuous database \cite{Polyak2015}.
The first version, published in June 11, 2013, contains 8 files with 194 columns each.
The next version of the database, published March 8, 2015, compiles all the data into a single file and reduces the number of columns to 54, marking a drastic change.
In addition, several columns changed the units with which they reported measurements.
While usage documentation, explaining the content and use of the data, accompanied each version, no records indicating what changed between versions were included.
A change log would be valuable guide with such drastic structural and content changes.
The third and most recent publication came in July 11, 2017, with no changes to the number of files or columns, but many new rows.

\section{Copper Dataset}

The Paragenetic Mode for Copper Minerals database became available through collaboration with the author's lab to create new methods of visualizing mineralogy relationships \cite{Morrison2016}.
The first version was collected June 8, 2016, with the update following soon after on August 8, 2016.
Major edits are fairly limited with only 16 column additions and 2 removals between the versions.
Value formats remain consistent from one version to the next, resulting in a much more condensed body of changes, making transitions more easily verifiable.
Compared to the Noble Gas data set, it provides a more stable data platform to implement the versioning model in Chapter \ref{ch:model}.
The data from this work is also more processing friendly, making it agreeable to automatic change log generation.

\section{GCMD Keywords}

The Global Change Master Directory (GCMD) is a metadata repository used by NASA to store records of its available data sets \cite{Miled:2001:GCM:372202.372324}.
They employ a set of keywords to make NASA Earth Science data sets searchable.
These words tag and label datasets into strictly defined categories \cite{GCMDKey}.
GCMD Keywords do not qualify as a standard web ontology since it does not constitute a class hierarchy.
The management team stored early versions of the keywords in Excel spreadsheets, and a centralized distribution system was not used until June 12, 2012.
The Key Management Service now serves the keywords directly in a variety of formats.
Each version of the keywords, encoded in RDF, were downloaded into separate files.
Only versions from June 12, 2012 and after were available, resulting in 9 version files.
Each keyword corresponds to a unique identifier, and when combined with a web namespace, resolves to a data description of the keyword.
Every identifier can be referred to per version by including the version's number at the web identifier's end, meaning that identifiers are consistent across versions.
The taxonomy uses the concepts \textit{skos:Broader} and \textit{skos:Narrower}, where skos refers to the Simple Knowledge Organization System ontology name space, to form a tree hierarchy \cite{skos}.
The tree's root is the keyword, "Science Keywords."
The data set provides an interesting study case due its long sequence of versions and ready use of linked data technology \cite{Stevens2016}.

\begin{table}
	\caption{List of versions available in the KMS.}
	\label{gcmd_table}
	\centering
	\begin{tabular}{|c|c|c|c|c|c|c|c|c|}
		\hline
		June 12, 2012 & 7.0 & 8.0 & 8.1 & 8.2 & 8.3 & 8.4 & 8.4.1 & 8.5 \\
		\hline
	\end{tabular}
\end{table}

\section{MBVL Classifications} \label{sec:MBVL}

The Marine Biodiversity Virtual Laboratory (MBVL) provides data and services for the study of marine biology with an integrative approach \cite{mbvl}.
In the application studied, a choice of algorithm and taxonomy pairings must be tested on a known population in order to estimate their performance with an unknown microbial population.
The original sequences belong only to the species listed in Table \ref{species_table}.
However, this data is not available to the author, and only the list of species are known, forming the first data set in this section.
These sequences are then grouped and classified by a specific taxonomy and algorithm pairing.
The workflow utilizes two taxonomies, the Ribosomal Database Project (RDP) and the Silva taxonomy.
Using these databases, the Species-level IdentificatioN of metaGenOmic amplicons (SPINGO) or the Global Alignment for Sequence Taxonomy (GAST) algorithms assign taxonomic ranks to each sequence.
This produces four data sets, each using the same grouping identifiers and having the same size in each group.
This means that the only difference between the data sets are the ranks assigned to each group.

\begin{table}
	\caption{List of species in the original population.}
	\label{species_table}
	\centering
	\setlength{\tabcolsep}{2pt}
	\begin{tabular}{|c|c|c|}
			\hline
			Acinetobacter baumannii & Actinomyces odontolyticus & Bacillus cereus \\
			Bacteroides vulgatus & Clostridium beijerinckii & Deinococcus radiodurans \\
			Enterococcus faecalis & Escherichia coli & Helicobacter pylori \\
			Lactobacillus gasseri & Listeria monocytogenes & Neisseria meningitidis\\
			Porphyromonas gingivalis & Propionibacterium acnes & Pseudomonas aeruginosa \\
			Rhodobacter sphaeroides & Staphylococcus aureus & Staphylococcus epidermidis\\
			Streptococcus agalactiae & Streptococcus mutans & Streptococcus pneumoniae \\
			\hline
	\end{tabular}
\end{table}
