%%%%%%%%%%%%%%%%%%%%%%%%%%%%%%%%%%%%%%%%%%%%%%%%%%%%%%%%%%%%%%%%%%%
%                                                                 %
%                            CHAPTER SIX                          %
%                                                                 %
%%%%%%%%%%%%%%%%%%%%%%%%%%%%%%%%%%%%%%%%%%%%%%%%%%%%%%%%%%%%%%%%%%%

\chapter{ONTOLOGY VERSIONING}

\section{Sea Ice Ontology}

A great many annotations were made to Version 2 of the Sea Ice Ontology.
Maintaining those notes after migration to the new version would provide great value to the project.
From the color code in Table \ref{ColorTable}, yellow and blue changes correspond directly to Modification and Addition transitions, respectively.
In theory, the green concepts would also qualify for a Modify categorization since they would likely have a new URI, meaning that an attribute exists between two versions and something has changed.
However, this would result in a product that is at least the size of the union of the two versions, greatly hindering the scalability of the approach.
The obvious solution would be to leave the attributes unlinked, as in the approach with the spreadsheet application where no change was detected.
The Invalidation change would cover concepts which do not have a mapping into Version 3.
Therefore, the remaining transition types, purple and red require more specific attention than a usual Modify.

\begin{table}
	\newcommand*{\thead}[1]{\multicolumn{1}{|c|}{\bfseries #1}}
	\centering
	\begin{tabular}{ | c | l |}
		\hline
		\textbf{Color} & \thead{Description} \\
		\hline
		Purple & Moved since the previous version of the ontology.\\
		\hline
		Green & Still the same.\\
		\hline
		Yellow & In the same place; but perhaps the name or definition have changed.\\
		\hline
		Red & Suggest changes to be made to both the ontologies and the nomenclature.\\
		\hline
		Blue & New concepts to be added to the ontologies.\\
		\hline
	\end{tabular}
	\label{ColorTable}
	\caption{Color code used by the concept maps made by Ruth Duerr during the planning phase of the Sea Ice Ontology's development.}
\end{table}

\section{GCMD Keywords}
GCMD Keywords do not qualify as a standard web ontology since it does not constitute a class hierarchy.
As a result, the addition, removal, or modification of any term within the keywords has a significant impact on the semantics of using a keyword to describe a data set.
More recent editions of the keywords, available through the Key Management Service (KMS), distributes they keywords in RDF format, and this allows them to be referenced through a unique web identifier.
The identifier drastically simplifies the attribution of changes between versions of the keyword list.
However, older editions of were stored and distributed using Excel spreadsheets.
This provides an interesting juxtaposition between versioning the spreadsheet as mentioned in a previous chapter and tracking the keyword changes.
Results from this activity enables data sets described with different versions of the GCMD Keywords to remain discoverable within the same system.

Better quality data for the early versions of the keywords needs to be acquired before scripts can make mappings to newer releases.
While easy to reference, the identifiers used to store the newer keywords are not immediately interpretable so further work needs to be done in order to form a mapping.
Once the versions have been mapped, the workflow for publishing the change data follows the same process as in the previous chapter.