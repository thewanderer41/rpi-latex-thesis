%%%%%%%%%%%%%%%%%%%%%%%%%%%%%%%%%%%%%%%%%%%%%%%%%%%%%%%%%%%%%%%%%%%
%                                                                 %
%                            CHAPTER SIX                          %
%                                                                 %
%%%%%%%%%%%%%%%%%%%%%%%%%%%%%%%%%%%%%%%%%%%%%%%%%%%%%%%%%%%%%%%%%%%

\chapter{ONTOLOGY VERSIONING}

Databases already maintain a part of versioning history with a transactional log.
However, they pose an interesting change in context compared to spreadsheets.
Often version comparisons occur between instances of spreadsheet files, but with databases, modifying transactions do not generate a new instance of the database.
Identifying a version would then need to adapt and link transactions to versions.
This can be done through query based citations as described by Proll and Rauber \cite{Proell2013}.
The transaction log also more specifically states the attribute involved, making detection of new attributes into the database more straightforward.
This addresses a particular concern in spreadsheet rows since their attributes have a tendency to be less consistent.
Since RRUFF already possesses an automatically, web accessible change log, the work in this area focuses primarily on deconstructing their code to hook in the concept model with RDFa.

\section{Sea Ice Ontology}

A great many annotations were made to Version 2 of the Sea Ice Ontology.
Maintaining those notes after migration to the new version would provide great value to the project.
From the color code in Table \ref{ColorTable}, yellow and blue changes correspond directly to Modification and Addition transitions, respectively.
In theory, the green concepts would also qualify for a Modify categorization since they would likely have a new URI, meaning that an attribute exists between two versions and something has changed.
However, this would result in a product that is at least the size of the union of the two versions, greatly hindering the scalability of the approach.
The obvious solution would be to leave the attributes unlinked, as in the approach with the spreadsheet application where no change was detected.
The Invalidation change would cover concepts which do not have a mapping into Version 3.
Therefore, the remaining transition types, purple and red require more specific attention than a usual Modify.

\begin{table}
	\newcommand*{\thead}[1]{\multicolumn{1}{|c|}{\bfseries #1}}
	\centering
	\begin{tabular}{ | c | l |}
		\hline
		\textbf{Color} & \thead{Description} \\
		\hline
		Purple & Moved since the previous version of the ontology.\\
		\hline
		Green & Still the same.\\
		\hline
		Yellow & In the same place; but perhaps the name or definition have changed.\\
		\hline
		Red & Suggest changes to be made to both the ontologies and the nomenclature.\\
		\hline
		Blue & New concepts to be added to the ontologies.\\
		\hline
	\end{tabular}
	\label{ColorTable}
	\caption{Color code used by the concept maps made by Ruth Duerr during the planning phase of the Sea Ice Ontology's development.}
\end{table}

