%%%%%%%%%%%%%%%%%%%%%%%%%%%%%%%%%%%%%%%%%%%%%%%%%%%%%%%%%%%%%%%%%%%
%                                                                 %
%                            CHAPTER SIX                          %
%                                                                 %
%%%%%%%%%%%%%%%%%%%%%%%%%%%%%%%%%%%%%%%%%%%%%%%%%%%%%%%%%%%%%%%%%%%

\chapter{ONTOLOGY VERSIONING}

The Global Change Master Directory is a repository to find Earth science data from NASA.  They employ a strict set of keywords to tag and label datasets in order to make them more searchable.  Version 1.0.0 of GCMD Keywords was published on April 24, 1995, and as of the time of writing, the most recent version of the keywords is 8.4.  As can be seen, the naming scheme of the versions changed since the first publication of the keywords.  In the initial scheme, each part of the decimal system represented a different level of the GCMD Keyword hierarchy.  When a change occured to a concept in a level of the hierarchy, the associated version number increments.

Since there exists an implementation of the GCMD Keywords in RDF, the URIs can be used as references for the Attributes in the concept model.  As mentioned in Reference 8, in order for a data consumer to understand how to use a new version of an ontology, they need not only understand what concepts are new and what concepts are old, but also how to map the old ontology onto the new ontology.  The mapping then informs the migration of Keyword labeling of datasets between versions.