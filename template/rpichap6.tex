%%%%%%%%%%%%%%%%%%%%%%%%%%%%%%%%%%%%%%%%%%%%%%%%%%%%%%%%%%%%%%%%%%%
%                                                                 %
%                            CHAPTER SIX                          %
%                                                                 %
%%%%%%%%%%%%%%%%%%%%%%%%%%%%%%%%%%%%%%%%%%%%%%%%%%%%%%%%%%%%%%%%%%%

\chapter{FUTURE WORK}\label{ch:future}

VersOn opens a number of different avenues for future versioning research.
Approaches to compressing modification statements when capturing change in spreadsheets need to be pursued.
The ability to study changes of a collection of data as a vector space was not pursued.
Data sets can now be studied by unique characteristics resulting from the development of new versioning versioning nomenclature to describe data sets.
Version analysis can be expanded by developing the ideas of change usage compression, vector analysis of change space, and the study of data set characteristics using new versioning nomenclature.

\section{Modification Compression}

\textbf{Modifications} had to include the same number of triples as cells modified in the Copper Minerals database.
Fewer triples can be used if changes to entire rows or columns can be summarized in a single statement.
Encoded change logs would see significant reductions in storage costs as a result.
Summarization by row or column may fail when a few cells do not change so a method to express no change in \gls{vo} must be pursued.
Using the smaller number of changes or unmodified values, at most half the length of a row or column in a spreadsheet worth of statements is necessary to describe a change to a row or column, respectively.

\subsection{Dynamic Change Logs}

Users selectively use portions of particularly expansive data sets to filter data down to their region of study. 
Tools can use the versioning model to identify pertinent sections of a large change log and parse out the extraneous entries.
Means to isolate change activities are necessary for users to determine the impact a new version has on the operation of their workflow.
The versioning graph can also contribute to the generation of unique change logs to accompany dynamically created data sets.
As mentioned in Section \ref{sec:grid}, users can dynamically aggregate and filter data sets to produce a new unique set of data, but doing so still requires tracking of differences from the original data set or sets.
Further work will need to be done determining requirements to automate change log creation for these data sets.

\section{Vector analysis of Change Space}

The three dimensional scatter plot shown in Figure \ref{EOL_AIM} show versions plotted in change space as a combination of \gls{AIM} counts.
All points are an absolute distance and direction from (0,0,0), meaning that each version can be represented as a vector.
Multiple versions could be added together to show the resultant change to a data set.
Further research is necessary to determine whether vector operations can be used to study version behavior.

\section{Nested Versioning graphs}

In Figure \ref{HCLSModel} and Figure \ref{hierarchy} use multiple levels to capture versions, but \gls{vo} only uses two.
Allowing \textbf{attributes} to also be \textbf{versions} means that a versioning graph can express variable granularity.
A graph captures a modification in a file of a data set managed by EOL.
The file would be an \textbf{attribute} of \textbf{versions} of the EOL data set, but a second graph would capture differences in values within the file, making the file a \textbf{version}.
The second graph could then be nested in the first by connecting the graphs at the file identifiers.

\section{Patterns of change behavior based on Change Classes}

The change classes of unversionable, irregular, and periodic organize data sets by different practices of versioning.
Further work is necessary to determine if there are versioning requirements specific to the type of change class, and determine if the classes exhibit specific patterns of change which could improve change data capture.
A particular versioning technique or tradition may not be appropriate for every class of data set.

\section{Database Context}

One area not explored by the work in this dissertation is the context of centralized databases.
While they resemble spreadsheets, centralized databases only have a single instance and use multiple tables which are routinely merged to answer queries.
The scripts and process used to version spreadsheets would not work on these databases since the data is not instanced.
The databases, however, use standardized add, delete, and modify commands which do map to the versioning model.
Work remains to be done in studying how these commands can be captured and output as a versioning graph instead of using the script to perform the comparison.

\section{Summary}

Future work remains open to a number of different pursuits.
Change logs can be shortened by discovering modifications occurring over an entire column which can be summarized in a single statement.
Version analysis can now leverage vector operations to study data set change.
Organizing data sets by versioning approach opens new avenues of evaluating the behavior and performance of versioning systems.
These approaches were left unexplored by the project's conclusion.