%%%%%%%%%%%%%%%%%%%%%%%%%%%%%%%%%%%%%%%%%%%%%%%%%%%%%%%%%%%%%%%%%%%
%                                                                 %
%                            CHAPTER SEVEN                        %
%                                                                 %
%%%%%%%%%%%%%%%%%%%%%%%%%%%%%%%%%%%%%%%%%%%%%%%%%%%%%%%%%%%%%%%%%%%

\chapter{FUTURE WORK}


While much has been expounded upon to develop the framework to conduct version distance calculations, nothing has been said as to how such a calculation would work.
The concept model specifically relates objects in a single direction from attributes in older versions to the corresponding one in newer editions.
Once values are assigned to each of the change types, a flux-like calculation can be performed to characterize the change moving from one side to another.
However, the resulting calculation may need to be more complex since the length of change logs is not standardized.
For example, a long list of small changes could over-shadow a few significant modifications.
Possible solutions could include sub-classing the change types to give a wider range of weights or normalizing the values across a range to give comparable results.

Generating results for database and keyword versioning fall largely on applying attributes to its data since the remaining workflow remains the same across contexts.