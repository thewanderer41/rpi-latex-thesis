%%%%%%%%%%%%%%%%%%%%%%%%%%%%%%%%%%%%%%%%%%%%%%%%%%%%%%%%%%%%%%%%%%%
%                                                                 %
%                            CHAPTER SEVEN                        %
%                                                                 %
%%%%%%%%%%%%%%%%%%%%%%%%%%%%%%%%%%%%%%%%%%%%%%%%%%%%%%%%%%%%%%%%%%%

\chapter{DISCUSSION}

\section{Model}

The model takes a version's generation from a provenance activity and disentangles it from the results of a versioning comparison.
The key lies in focusing on the version's attributes in order to explain changes between the version.
Compared to PROV's derivation property, involving 3 triples, or it's qualified form, which involves 5 triples, the versioning model uses 9 triples in the worst case with a single modification relationship.
In addition, the space usage grows linearly with respect to the number of changes while the PROV relation stays constant at 3 or 5.
The trade-off is in the extra machine-computable characterization of changes which would normally be contained in a solely human-readable change log.

In the case where existing ontologies are repurposed to implement the model, both PROV and Schema.org ontologies view versioning as an active practice rather than a study.
In both, an agent actively interacts with a set or collection.
This model follows a different view of versioning where the relationships observed result from state-based properties, not an actor.

This model combines the observations that the core operations implemented by versioning systems is add, invalidate, and modify.

\section{Implementation}

The versioning process breaks down to three formal steps.
The first activity verifies that the objects being compared are actually versions of each other.
This exercise is often left out of details in practice since a data producer is often fairly certain as to the state of their versions.
However, this establishes the foundation and validity of further actions taken to version the objects.
This ensures that a mapping can be performed and will be meaningful.
The next step is generating the mapping to identify addition, invalidation, and modification relationships.
The resultant mapping in spreadsheet comparisons followed very similar rules, but when looking at the MBVL dataset, the definitions were changed to achieve a specific goal.
The final step involves publishing the change information using the mapping.
In this thesis, the resulting product is published into a versioning graph.

One of the desired contributions was to study the possibility of a machine readable change log.
In this implementation, the number of triples necessary to implement the model significantly impairs the ability of the log to remain human readable.
One drawback is that the modification of an entire column would result in multiple entries equal to the number of rows in the table.
JSON-LD proves to be a better mechanism for encoding than RDFa due to the data storage goal in the change log.

With regards to the resulting versioning graph, it highlights an important concept of state-based properties.
An activity does not actively impose change relationships between two versions.
This also means that connections will also develop with prior versions.
An activity based view of versioning captures only a single step and loses the larger view of continued change implications between versions farther than one step apart.

\section{Version Identification}

Version identifiers provide a very important role in communicating change information because users see it first.
However, their application is subjective as can be seen in the case of GCMD Keywords.
The producers created a change in their data set which they did not report in their change log or capture in their identifier.
By connecting indicators to change measures such as the versioning model of this thesis, they better convey the change within a data set.
A poor understanding of the amount of change between two versions can lead to flawed expectations in migrating across them.
This can lead to extra or unexpected costs in using the newer data set.

Another thing to consider is that counters in magnitudes by themselves may not characterize data set change well.
In Figure \ref{GCMDC2}, the yellow line indicates the total changes made to the data set, performing a similar function as the major/minor/revision version identifier.
However, breaking up the changes into types reveals the dominant contribution of additions to the data set.
This understanding of the data's behavior cannot be revealed with a three number dot decimal identifier system.
