%%%%%%%%%%%%%%%%%%%%%%%%%%%%%%%%%%%%%%%%%%%%%%%%%%%%%%%%%%%%%%%%%%%
%                                                                 %
%                            ABSTRACT                             %
%                                                                 %
%%%%%%%%%%%%%%%%%%%%%%%%%%%%%%%%%%%%%%%%%%%%%%%%%%%%%%%%%%%%%%%%%%%

\specialhead{ABSTRACT}

Data sets invariably require versioning systems to manage changes due to an imperfect collection environment.
While importance grows, versioning discussion remains imprecise, lacking standardization or formal specifications.
Many works tend to define versions around examples and local characteristics but lack a broader foundation.
This imprecision results in a reliance on change brackets and dot-decimal identifiers without quantitative measures to justify their application.
No difference exists between the versioning practices of a group which updates their data regularly and a group which adds many new files but rarely replaces them.
This work attempts to improve discussion by capturing version relationships into a linked data model, taking inspiration from provenance models that incorporate versioning concepts such as PROV and PAV.
The model captures addition, invalidation, and modification relationships between versions to provide change log-like characterization of the differences.
This approach demonstrated increased expressibility of change interactions, but encountered issues with space scalability.
The model's generation also revealed a four step process to conduct versioning: validation, mapping, computation, and publishing.
Quantifying these changes also provided a numerical basis for evaluating the GCMD Keywords taxonomy's adopted identification scheme.
It also demonstrates the ability of versioning methods to actively influence scientific designs through performance assessment.