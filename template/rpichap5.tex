%%%%%%%%%%%%%%%%%%%%%%%%%%%%%%%%%%%%%%%%%%%%%%%%%%%%%%%%%%%%%%%%%%%
%                                                                 %
%                            CHAPTER FIVE                         %
%                                                                 %
%%%%%%%%%%%%%%%%%%%%%%%%%%%%%%%%%%%%%%%%%%%%%%%%%%%%%%%%%%%%%%%%%%%

\chapter{DATABASE VERSIONING}

The framework for communicating change information resembles the spreadsheet context.  However, the main difference is the method to refer to the Attributes as they are mapped across versions.  Database tables do not have strict identifiers for rows and columns as spreadsheets do.  Row identification relies on indexed keys to uniquely identify entries, but these entries can also be arranged and presented in different orderings depending on the queries used to view the database.  From Reference 24, elements within the database do not have to be digitally organized in the same way that it is physically presented.

Databases are also meant to be kept online for extended periods of time.  Spreadsheets must be entirely republished in order for a change to the data to be made public.  Changes in a database are therefore more sparse in between each version.  In order to make changelogs more comprehensive, they can describe changes by day or other increment of time or describe the latest particular subset of time.

Once the changes within the database are encoded into the versioning conceptual model, publishing the changes can follow the methods detailed in Chapter~\ref{ch:spreadsheet}