%%%%%%%%%%%%%%%%%%%%%%%%%%%%%%%%%%%%%%%%%%%%%%%%%%%%%%%%%%%%%%%%%%%
%                                                                 %
%                            CHAPTER FIVE                         %
%                                                                 %
%%%%%%%%%%%%%%%%%%%%%%%%%%%%%%%%%%%%%%%%%%%%%%%%%%%%%%%%%%%%%%%%%%%

\chapter{ANALYSIS}

\section{Introduction}

Implementing the versioning model yielded results more complicated than the simple model expected.
While the model addresses difficulties in other linked data approaches, it requires many more triples to express the relationship.
The scalability created space issues with encoded change logs, especially in JSON-LD.
RDFa also proved to be a more restrictive structured data method than expected.
The implementation required multiple attributes per \textbf{modification} to accommodate both row and column \textbf{attributes} associating with a cell.
There were discrepancies between GCMD Keyword version identifiers and the change detected within the data set.
Finally, the versioning model was used not to document sequential versions but to compare the results of different species classifiers.

\section{Model}

The versioning model's development began with an expectation that versions would be sequential.
The Marine Biodiversity Virtual Laboratory (MBVL) data set demonstrated a case where four data sets were not related by temporal sequence.
One is not a transformation of another since we are studying the effects of changing the taxonomy or algorithm.
Additionally, since we do not know which version is the best, we cannot consider any data set as an update of the others.
Finally, no entity preexisted as the data sets resulted from an ongoing analysis and further steps have not been developed.
As a result, the current definition of \textit{prov:wasDerivedFrom} would not be able to capture the relationship between these data sets.
The model improves upon expressing versions in linked data by focusing on the differences between objects rather than the sequence.
The model takes inspiration from \textit{schema:UpdateAction} by dividing up the \textbf{changes} into three forms, but improves upon it by adopting the provenance model's transition from one object to the next.
The resulting forms diverge from Schema.org's context of an agent acting upon an object.

The reason \textit{prov:Generation} and \textit{prov:Invalidation} are not used is because they expect an activity to act upon an object.
It is not generally true that an action actively adds or removes an object's attribute from in the left-hand version to produce the right-hand revision.
That assumption minimizes the ability to conduct versioning comparisons between objects that are not sequentially adjacent.
The PROV concepts also have a property pointing towards the responsible activity which is assumed to be the immediately preceding activity.
The assumption fails to consider the case where a change propagates further changes downstream, generating or invalidating the current object
The versioning model avoids confusion by only considering the versions and their differences.

\section{Implementation}

\subsection{Scalability}

The versioning model breaks up a revision into constituent changes, acting upon different attributes of the version.
Other ontologies use a single property to relate versions.
While it is more specific, the VersOn implementation encounters scalable space consumption problems.
PROV only requires 3 to 5 triples in order to make a \textit{prov:wasRevisionOf} statement.
This model uses 9 triples for a \textit{vo:ModifyChange} and 7 to encode \textit{vo:AddChange} and \textit{vo:InvalidateChange}.
An implementation of the model, therefore, has space complexity of \(O(7M+5(A+I))\) since declaring version objects takes a constant two statements.
However, a similar structure can be achieved using \textit{prov:wasDerivedFrom} to replace modifications and \textit{schema:AddAction} and \textit{schema:DeleteAction} to replace additions and invalidations.
The resulting space complexity is \(O(7M+3A+5I)\).
This is fairly similar with additions seeing a reduction since the left-hand version no longer contributes to the \textit{schema:AddAction}.
Thus the primary benefit of using this model comes from semantics.

\subsection{Structured Data and the Model}

While machine-readable change logs have always been a desired goal of this dissertation work, their requirements diverged from the versioning model's needs.
The model, as a result, leverages very little from visible content on the change log.
Symmetric representation in the log also made encoding the graph using RDFa challenging without explicitly defining the whole graph in invisible span tags.
Adherence with a log oriented approach would also likely have reduced the number of statements needed to form the versioning graph.
The resulting ontology would likely be a collection of properties and concepts to use in annotating a document.

The current model construction provides great flexibility for version and distance capture.
The model adapts to multiple attributes smoothly.
Greater adherence to structured data adoption may need to come in the form of graph simplification or metered release of new editions to ensure that change logs do not grow too large.

\section{Distance Measure}

As mentioned in Section \ref{sec:usecase}, a version model provides the framework, provenance models provide the context, and change logs fill in the gap between versions.
Change logs, therefore, provide the most substance to quantify the distance between versions.
The automated log generation additionally ensures this by including all differences into the change log.
Anything unmapped remains the same between versions and does not contribute towards the distance.
While MBVL demonstrated a case where domain knowledge could be added to the versioning graph and provide context for distances, other applications may not demonstrate the same amount of uniformity within changes.
More domain information and reasoning may be necessary to determine if one add change significantly more impactful than others in a versioning graph.

\section{Summary}

The versioning model uses expanded semantics to better capture the differences between versions.
When implemented in JSON-LD, the versioning graph integrates well with text change logs, but it must address scalability issues with more volatile data sets.
The model's construction allows multiple versions to be linked together into a single graph, but graphs with four or more versions may have problems with discontinuous attributes.
The implementation was not able to provide evidence linking change counts to version identifiers due to strong disagreement with GCMD Keywords version 8.5.
The results do indicate that version identifiers need better quantitative support.
The MBVL results also demonstrate that the versioning model can provide comparisons in more contexts than documentation.
