%%%%%%%%%%%%%%%%%%%%%%%%%%%%%%%%%%%%%%%%%%%%%%%%%%%%%%%%%%%%%%%%%%%
%                                                                 %
%                            APPENDICES                           %
%                                                                 %
%%%%%%%%%%%%%%%%%%%%%%%%%%%%%%%%%%%%%%%%%%%%%%%%%%%%%%%%%%%%%%%%%%%

\appendix    % This command is used only once!
%\addcontentsline{toc}{chapter}{APPENDICES}             %toc entry  or:
\addtocontents{toc}{\parindent0pt\vskip12pt APPENDICES} %toc entry, no page #

\chapter{NOBLE GAS CHANGE LOG GENERATOR VERSION 1 TO 2}
\begin{minted}[linenos, frame=lines, baselinestretch=1.2, breaklines]{python}
from os.path import join, dirname, abspath, isfile
from os import sep as separator
import xlrd, sys, json
import glob
import re


def index_convert(index1):
if index1 < 17:
return index1
elif index1 < 24:
return index1+1
elif index1 < 26:
return 26+(3*(index1-24))
elif index1 < 28:
return 44+(3*(index1-26))
elif index1 < 32:
return 53+(9*(index1-28))
elif index1 < 36:
return 88+(2*(index1-32))
elif index1 < 38:
return 95+(4*(index1-36))
elif index1 < 41:
return 102+(2*(index1-38))
elif index1 < 43:
return 112+(2*(index1-41))
elif index1 == 43:
return 172
elif index1 < 50:
return 174+(3*(index1-44))
elif index1 < 54:
return 191+(index1-50)
else:
print 'Error: Out of bounds'
return -1

def test_alignment():
for i in range(0, 54):
print 'version2: {:5} version1: {:5}'.format(i, index_convert(i))

def compare_print(mode, key, val1, val2, v1_file, v1_index = 0, v2_index = 0, changelog = None):
if changelog:
if mode == 'r':
out = u'''        <tr  about="Change{}{}" typeof="vo:ModifyChange">
<td align="right" rev="vo:Undergoes" resource="v1:Attribute{}{}v1" typeof="vo:Attribute">{:2}({})</td>
<td property="vo:resultsIn" resource="v2:Attribute{}{}v2" typeof="vo:Attribute">{:2}</td>
<td>{:>10}</td>
<td>{:>10}</td>
<span about="Version1" property="vo:hasAttribute" resource="v1:Attribute{}{}v1"></span>
<span about="Version2" property="vo:hasAttribute" resource="v2:Attribute{}{}v2"></span>
</tr>\n'''.format(key, v2_index, key, v1_index, v1_index, v1_file, key, v2_index, v2_index, val1, val2, key, v1_index, key, v2_index)
elif mode == 'j':
out = u'''        <tr  id="ModifyChange{}{}">
<td align="right">{:2}({})</td>
<td>{:2}</td>
<td>{:>10}</td>
<td>{:>10}</td>
<script type="application/ld+json">\n'''.format(key, v2_index, v1_index, v1_file, v2_index, val1, val2)
elif mode == 't':
out = u"{:2}({})\t{:2}\t{:>10}\t{:>10}\n".format(v1_index, v1_file, v2_index, val1, val2)
elif mode == 'u':
out = u"""<http://example.com/NG/Version1> vo:hasAttribute <http://example.com/NG/Version1/%s> ;
vo:hasAttribute <http://example.com/NG/Version1/Column%i> .
<http://example.com/NG/Version1/%s> a vo:Attribute ;
vo:undergoes <http://example.com/Changelog#ModifyChange%s%i> .
<http://example.com/NG/Version1/Column%i> a vo:Attribute ;
vo:undergoes <http://example.com/Changelog#ModifyChange%s%i> .
<http://example.com/Changelog#ModifyChange%s%i> a vo:ModifyChange ;
vo:resultsIn <http://example.com/NG/Version2/%s> ;
vo:resultsIn <http://example.com/NG/Version2/Column%i> .
<http://example.com/NG/Version2> vo:hasAttribute <http://example.com/NG/Version2/%s> ;
vo:hasAttribute <http://example.com/NG/Version2/Column%i> .

"""%(key, v1_index, key, key, v2_index, v1_index, key, v2_index, key, v2_index, key, v2_index, key, v2_index)
changelog.write(out.encode('utf8'))
if mode == 'j':
json1 = {
"@context":context,
"@type":"vo:Attribute" ,
"@id":"".join(["http://ngdb.com/v1/Attribute", key, str(v1_index)]) ,
"label":key ,
"undergoes":"".join([host, "ModifyChange", key, str(v2_index)]) ,
"@reverse" :    { "hasAttribute" : "Version1" }
}
json2 = {
"@context":context,
"@type":"vo:ModifyChange",
"@id":"".join([host, "ModifyChange", key, str(v2_index)]) ,
"resultsIn":"".join(["http://ngdb.com/v2/Attribute", key, str(v2_index)])
}
json3 = {
"@context":context,
"@type":"vo:Attribute" ,
"@id":"".join(["http://ngdb.com/v2/Attribute", key, str(v2_index)]) ,
"label":key ,
"@reverse" :    { "hasAttribute" : "Version2" }
}
json.dump([json1, json2, json3], changelog, indent=4, sort_keys=True)
changelog.write('''
</script>
</tr>
''')
else:
print '{:5}  version1: {:10} version2: {:10}'.format(key, val1, val2)

labels = {17:"SAMPLING - DEPTH - >,<",
25:"[He] - ppm - >,<", 27:"[He] - ppm - err", 28:"[He] - mkcc/ - >,<", 30:"[He] - mkcc/ - err", 31:"[He] - mol/ - >,<", 32:"[He] - mol/ - L H2O", 33:"[He] - mol/ - err",
34:"[He+Ne] - ppm - >,<", 35:"[He+Ne] - ppm", 36:"[He+Ne] - ppm - err", 37:"[He+Ne] - mkcc/ - >,<", 38:"[He+Ne] - mkcc/ - g H2O", 39:"[He+Ne] - mkcc/ - err", 40:"[He+Ne] - mol/ - >,<", 41:"[He+Ne] - mol/ - L H2O", 42:"[He+Ne] - mol/ - err",
43:"[Ne] - ppm - >,<", 45:"[Ne] - ppm - err", 46:"[Ne] - mkcc/ - >,<", 48:"[Ne] - mkcc/ - err", 49:"[Ne] - mol/ - >,<", 50:"[Ne] - mol/ - L H2O", 51:"[Ne] - mol/ - err",
52:"[20Ne] - ppm - >,<", 54:"[20Ne] - ppm - err", 55:"[20Ne] - mkcc/ - >,<", 56:"[20Ne] - mkcc/ - g H2O", 57:"[20Ne] - mkcc/ - err", 58:"[20Ne] - mol/ - >,<", 59:"[20Ne] - mol/ - L H2O", 60:"[20Ne] - mol/ - err",
61:"[Ar] - ppm - >,<", 63:"[Ar] - ppm - err", 64:"[Ar] - mkcc/ - >,<", 65:"[Ar] - mkcc/ - g H2O", 66:"[Ar] - mkcc/ - err", 67:"[Ar] - mol/ - >,<", 68:"[Ar] - mol/ - L H2O", 69:"[Ar] - mol/ err",
70:"[Kr] - ppm - >,<", 72:"[Kr] - ppm - err", 73:"[Kr] - mkcc/ - >,<", 74:"[Kr] - mkcc/ - g H2O", 75:"[Kr] - mkcc/ - err", 76:"[Kr] - mol/ - >,<", 77:"[Kr] - mol/ - L H2O", 78:"[Kr] - mol/ err",
79:"[Xe] - ppm - >,<", 81:"[Xe] - ppm - err", 82:"[Xe] - mkcc/ - >,<", 83:"[Xe] - mkcc/ - g H2O", 84:"[Xe] - mkcc/ - err", 85:"[Xe] - mol/ - >,<", 86:"[Xe] - mol/ - L H2O", 87:"[Xe] - mol/ err",
89:"3He/4He - (R/Ra)me - err", 91:"3He/4He - (R/Ra)corr - err", 93:"3He/4He - Rme - E-8 - err", 96:"3He/4He - Rcorr - E-8 - err", 97:"Rank",
98:"He/Ne - >,<", 100:"He/Ne - >,<", 101:"4He/20Ne - >,<", 103:"4He/20Ne - err", 105:"20Ne/22Ne - err", 107:"21Ne/22Ne - (xE-2) - err", 108:"21Ne/20Ne", 109:"21Ne/20Ne - err",
110:"22Ne/20Ne", 111:"22Ne/20Ne - err", 113:"38Ar/36Ar - err", 115:"40Ar/36Ar - err", 116:"delta(40Ar)rad", 117:"delta(40Ar)rad - err",
118:"He/Ar - He/ - /Ar(air) - >,<", 119:"He/Ar - He/ - /Ar(air)", 120:"He/Ar - He/ - /Ar(air) - err", 121:"He/Ar - 4He/ - /36Ar - >,<", 122:"He/Ar - 4He/ - /36Ar", 123:"He/Ar - 4He/ - /36Ar - err",
124:"He/Ar - 4He/ - /40Ar(air) - >,<", 125:"He/Ar - 4He/ - /40Ar(air)", 126:"He/Ar - 4He/ - /40Ar(air) - err",
127:"f(He)=(He/Ar)s/(He/Ar)air - >,<", 128:"f(He)=(He/Ar)s/(He/Ar)air", 129:"f(He)=(He/Ar)s/(He/Ar)air - err",
130:"Ne/Ar - Ne/ - /Ar(air) - >,<", 131:"Ne/Ar - Ne/ - /Ar(air)", 132:"Ne/Ar - Ne/ - /Ar(air) - err", 133:"Ne/Ar - 20Ne/ - /36Ar - >,<", 134:"Ne/Ar - 20Ne/ - /36Ar", 135:"Ne/Ar - 20Ne/ - /36Ar - err",
136:"Ne/Ar - 20Ne/ - /40Ar(air) - >,<", 137:"Ne/Ar - 20Ne/ - /40Ar(air)", 138:"Ne/Ar - 20Ne/ - /40Ar(air) - err", 139:"Ne/Ar - 22Ne/ - /36Ar - >,<", 140:"Ne/Ar - 22Ne/ - /36Ar", 141:"Ne/Ar - 22Ne/ - /36Ar - err",
142:"Ne/Ar - 22Ne/ - /40Ar(air) - >,<", 143:"Ne/Ar - 22Ne/ - /40Ar(air)", 144:"Ne/Ar - 22Ne/ - /40Ar(air) - err",
145:"f(Ne)=(Ne/Ar)s/(Ne/Ar)air - >,<", 146:"f(Ne)=(Ne/Ar)s/(Ne/Ar)air", 147:"f(Ne)=(Ne/Ar)s/(Ne/Ar)air - err",
148:"Kr/Ar - Kr/ - /Ar(air) - >,<", 149:"Kr/Ar - Kr/ - /Ar(air)", 150:"Kr/Ar - Kr/ - /Ar(air) - err", 151:"Kr/Ar - 84Kr/ - /36Ar - >,<", 152:"Kr/Ar - 84Kr/ - /36Ar", 153:"Kr/Ar - 84Kr/ - /36Ar - err",
154:"Kr/Ar - 84Kr/ - /40Ar(air) - >,<", 155:"Kr/Ar - 84Kr/ - /40Ar(air)", 156:"Kr/Ar - 84Kr/ - /40Ar(air) - err",
157:"f(Kr)=(Kr/Ar)s/(Kr/Ar)air - >,<",158:"f(Kr)=(Kr/Ar)s/(Kr/Ar)air", 159:"f(Kr)=(Kr/Ar)s/(Kr/Ar)air - err",
160:"Xe/Ar - Xe/ - /Ar(air) - >,<", 161:"Xe/Ar - Xe/ - /Ar(air)", 162:"Xe/Ar - Xe/ - /Ar(air) - err", 163:"Xe/Ar - 132Xe/ - /36Ar - >,<", 164:"Xe/Ar - 132Xe/ - /36Ar", 165:"Xe/Ar - 132Xe/ - /36Ar - err",
166:"Xe/Ar - 132Xe/ - /40Ar(air) - >,<", 167:"Xe/Ar - 132Xe/ - /40Ar(air)", 168:"Xe/Ar - 132Xe/ - /40Ar(air) - err",
169:"f(Xe)=(Xe/Ar)s/(Xe/Ar)air - >,<", 170:"f(Xe)=(Xe/Ar)s/(Xe/Ar)air", 171:"f(Xe)=(Xe/Ar)s/(Xe/Ar)air - err",
173:"H2 - >,<", 175:"H2 - ppm - err", 176:"O2 - >,<", 178:"O2 - ppm - err", 179:"N2 - >,<", 181:"N2 - ppm - err", 182:"CO2 - >,<", 184:"CO2 - ppm - err", 185:"CH4 - >,<", 187:"CH4 - ppm - err",
188:"H2S - >,<", 190:"H2S - ppm - err"}

context = "https://orion.tw.rpi.edu/~blee/provdist/GCMD/VO.jsonld"
host = "http://orion.tw.rpi.edu/~blee/provdist/NobleGas/changelog_json.html#"
#test_alignment()


#print v2_row[0].value
#print indicator_map[v2_row[0].value]

#v1_workbook = xlrd.open_workbook(v1_file)
#v1_sheet = v1_workbook.sheet_by_index(0)
#v1_row = v1_sheet.row(4)

def write_modify(r1, r2, workbook, f_out, mode):
if mode == 'r':
out = u'''  <div about="Version1" rel="vo:hasAttribute">
<div resource="v2:%s" typeof="vo:Attribute">
<span style="font-weight:bold" property="http://www.w3.org/2000/01/rdf-schema#label">%s</span>
<table rel="vo:Undergoes">
'''%(r2[0].value, r2[0].value)
elif mode == 'j':
out = u'''
<div about="v2:%s">
<span style="font-weight:bold" property="http://www.w3.org/2000/01/rdf-schema#label">%s</span>
<table>
'''%(r2[0].value, r2[0].value)
elif mode == 't':
out = u"%s\n"%(r2[0].value)
elif mode == 'u':
out = u""

if mode == 'r' or mode == 'j':
out = out+'''        <tr>
<th>Column v1</th>
<th>Column v2</th>
<th>Version 1</th>
<th>Version 2</th>
</tr>\n'''
elif mode == 't':
out = out+"Column v1\tColumn v2\tVersion 1\tVersion 2\n"
f_out.write(out.encode('utf8'))
#print '# Searching...'
#print '# Comparing...'
for i in range(0,54):
if r2[i].value != r1[index_convert(i)].value:
#compare_print(j, v1_row[index_convert(j)].value, v2_row[j].value)
compare_print(mode, r2[0].value, r1[index_convert(i)].value, r2[i].value, workbook.split('/')[-1], index_convert(i), i, f_out)
if mode == 'r' or mode == 'j':
f_out.write('  </table></div><br>\n')
elif mode == 't' or mode == 'u':
f_out.write("\n")

def write_removed(v2, col, row, f_out, mode):
if mode == 'r' or mode == 'j':
f_out.write('''
<h3>Columns invalidated by %s</h3>
<table about="Version2">
'''%(v2.split('/')[-1]))
elif mode == 't':
f_out.write("\nColumns invalidated by %s\n"%(v2.split('/')[-1]))

print "Removed Column"
for i in col:
v1_value = labels.get(i, "")
if mode == 'r':
out = u'''        <tr resource="InvlidateChange%i" rev="vo:invalidatedBy" typeof="vo:InvalidateChange">
<td resource="Attribute%i" rev="vo:Undergoes" typeof="vo:Attribute">%i</td>
<td about="Attribute%i" property="http://www.w3.org/2000/01/rdf-schema#label">%s</td>
<span about="Version1" property="vo:hasAttribute" resource="Attribute%i"/>
</tr>
'''%(i, i, i, i, v1_value, i)
elif mode == 'j':
out = u'''        <tr id="InvlidateChange%i" about="InvlidateChange%i">
<td>%i</td>
<td>%s</td>
<script type="application/ld+json">
'''%(i, i, i, v1_value)
elif mode == 't':
out = u"%i\t%s\n"%(i, v1_value)
elif mode == 'u':
out = u"""<http://example.com/NG/Version1> vo:hasAttribute <http://example.com/NG/Version1/Column%s> .
<http://example.com/NG/Version1/%s> vo:undergoes <http://example.com/Changelog#InvalidateChange%i> .
<http://example.com/Changelog#InvalidateChange%i> a vo:InvalidateChange ;
vo:invalidatedBy <http://example.com/NG/Version2> .

"""%(i, i, i, i)
f_out.write(out.encode('utf8'))
if mode == 'j':
json1 = {
"@context":context,
"@type":"vo:Attribute" ,
"@id":"".join(["http://ngdb.com/v1/Attribute", str(i)]) ,
"label": v1_value,
"undergoes":"".join([host, "InvalidateChange", str(i)]) ,
"@reverse" :    { "hasAttribute" : "Version1" }
}
json2 = {
"@context":context,
"@type":"vo:InvalidateChange" ,
"@id": "".join([host, "InvalidateChange", str(i)]) ,
"invalidatedBy"  :   "Version2"
}
json.dump([json1, json2], f_out, indent=4, sort_keys=True)
f_out.write('''
</script>
</tr>
''')


if mode == 'r' or mode == 'j':
f_out.write('''      </table>
<h3>Rows invalidated by %s</h3>
<table about="Version2">
'''%(v2.split('/')[-1]))
elif mode == 't':
f_out.write("\nRows invalidated by %s\n"%(v2.split('/')[-1]))
elif mode == 'u':
f_out.write("\n")

print "Removed Row"
workbook_name = ''
for i, j in sorted(row, key=lambda x: x[0]):
if workbook_name != j:
workbook_name = j
v1_workbook = xlrd.open_workbook(workbook_name)
v1_sheet = v1_workbook.sheet_by_index(0)
v1_col = v1_sheet.col(0)
v1_col = [k.value for k in v1_col]
v1_index = v1_col.index(i)
if mode == 'r':
out = u'''        <tr resource="InvlidateChange%i" rev="vo:invalidatedBy" typeof="vo:InvalidateChange">
<td resource="Attribute%i" rev="vo:Undergoes" typeof="vo:Attribute">%i(%s)</td>
<td about="Attribute%i" property="http://www.w3.org/2000/01/rdf-schema#label">%s</td>
<span about="Version1" property="vo:hasAttribute" resource="Attribute%i"/>
</tr>
'''%(v1_index, v1_index, v1_index, workbook_name.split('/')[-1], v1_index, i, v1_index)
elif mode == 'j':
out = u'''        <tr id="InvlidateChange%i" about="InvlidateChange%i">
<td>%i(%s)</td>
<td>%s</td>
<script type="application/ld+json">
'''%(v1_index, v1_index, v1_index, workbook_name.split('/')[-1], i)
elif mode == 't':
out = u"%i(%s)\t%s\n"%(v1_index, workbook_name.split('/')[-1], i)
elif mode == 'u':
out = u"""<http://example.com/NG/Version1> vo:hasAttribute <http://example.com/NG/Version1/%s> .
<http://example.com/NG/Version1/%s> vo:undergoes <http://example.com/Changelog#InvalidateChange%s> .
<http://example.com/Changelog#InvalidateChange%s> a vo:InvalidateChange ;
vo:invalidatedBy <http://example.com/NG/Version2> .

"""%(i, i, i, i)
f_out.write(out.encode('utf8'))
if mode == 'j':
json1 = {
"@context":context,
"@type":"vo:Attribute" ,
"@id":"".join(["http://ngdb.com/v1/Attribute", str(i)]) ,
"label": str(i),
"undergoes":"".join([host, "InvalidateChange", str(i)]) ,
"@reverse" :    { "hasAttribute" : "Version1" }
}
json2 = {
"@context":context,
"@type":"vo:InvalidateChange" ,
"@id": "".join([host, "InvalidateChange", str(i)]) ,
"invalidatedBy"  :   "Version2"
}
json.dump([json1, json2], f_out, indent=4, sort_keys=True)
f_out.write('''
</script>
</tr>
''')
if mode == 'r' or mode == 'j':
f_out.write('''      </table>

''')
elif mode == 't' or mode == 'u':
f_out.write("\n")


def write_added(v2, col, row, f_out, mode):
if mode == 'r' or mode == 'j':
f_out.write('''
<h3>Columns added by %s</h3>
<table about="Version1" rel="vo:absentFrom">
'''%(v2.split('/')[-1]))
elif mode == 't':
f_out.write("\nColumns added by %s\n\n"%(v2.split('/')[-1]))

print "Added Column"
for i in col:
print i#, v2_value
if mode == 'r':
f_out.write('''        <tr about="AddChange%i" typeof="vo:AddChange">
<td property="vo:resultsIn" resource="Attribute%i" typeof="vo:Attribute">%i</td>
<td about="Attribute%i" property="http://www.w3.org/2000/01/rdf-schema#label"></td>
<span about="Version2" property="vo:hasAttribute" resource="Attribute%i"/>
</tr>
'''%(i, i, i, i, i))
elif mode == 'j':
f_out.write('''        <tr id="AddChange%i" about="v2:Attribute%i">
<td>%i</td>
<td></td>
<script type="application/ld+json">
'''%(i, i, i))
json1 = {
"@context":context,
"@type":"vo:AddChange" ,
"@id": "".join([host, "AddChange", str(i)]) ,
"resultsIn" :   "".join([ "http://ngdb.com/v2/Attribute", str(i)]),
"@reverse"  :   { "absentFrom": "Version1" }
}
json2 = {
"@context":context,
"@type":"vo:Attribute" ,
"@id":"".join(["http://ngdb.com/v2/Attribute", str(i)]) ,
"label":"" ,
"@reverse" :    { "hasAttribute" : "Version2" }
}
json.dump([json1, json2], f_out, indent=4, sort_keys=True)
f_out.write('''
</script>
</tr>
''')
elif mode == 't':
f_out.write("%i\t\n"%(i))	
elif mode == 'u':
f_out.write("""<http://example.com/NG/Version1> vo:absentFrom <http://example.com/Changelog#AddChange%i> .
<http://example.com/Changelog#AddChange%i> a vo:AddChange ;
vo:resultsIn <http://example.com/NG/Version2/Column%s> .
<http://example.com/NG/Version2> vo:hasAttribute <http://example.com/NG/Version2/Column%s> .

"""%(i, i, i, i))
if mode == 'r' or mode == 'j':
f_out.write('''      </table>
<h3>Rows added by %s</h3>
<table about="Version1" rel="vo:absentFrom">
'''%(v2.split('/')[-1]))
elif mode == 't':
f_out.write("\nRows added by %s\n\n"%(v2.split('/')[-1]))
elif mode == 'u':
f_out.write("\n")

print "Added Row"
for i, j in row:#i is the id, j is the file
if mode == 'r':	                #print i, v2_sheet.cell(i,0).value
out = u'''        <tr about="AddChange%i" typeof="vo:AddChange">
<td property="vo:resultsIn" resource="Attribute%i" typeof="vo:Attribute">%i</td>
<td about="Attribute%i" property="http://www.w3.org/2000/01/rdf-schema#label">%s</td>
<span about="Version2" property="vo:hasAttribute" resource="Attribute%i"/>
</tr>
'''%(i, i, i, i, j, i)
elif mode == 'j':
out = u'''        <tr id="AddChange%i" about="v2:Attribute%i">
<td>%i</td>
<td property="http://www.w3.org/2000/01/rdf-schema#label">%s</td>
<script type="application/ld+json">
'''%(i, i, i, j)
elif mode == 't':
out = u"%i\t%s\n"%(i, j)
elif mode == 'u':
out = u"""<http://example.com/NG/Version1> vo:absentFrom <http://example.com/Changelog#AddChange%i> .
<http://example.com/Changelog#AddChange%i> a vo:AddChange ;
vo:resultsIn <http://example.com/NG/Version2/%s> .
<http://example.com/NG/Version2> vo:hasAttribute <http://example.com/NG/Version2/%s> .

"""%(i, i, i, i)
f_out.write(out.encode('utf8'))
if mode == 'j':
json1 = {
"@context":context,
"@type":"vo:AddChange" ,
"@id": "".join([host, "AddChange", str(i)]) ,
"resultsIn" :   "".join([ "http://ngdb.com/v2/Attribute", str(i)]),
"@reverse"  :   { "absentFrom": "Version1" }
}
json2 = {
"@context":context,
"@type":"vo:Attribute" ,
"@id":"".join(["http://ngdb.com/v2/Attribute", str(i)]) ,
"label": j ,
"@reverse" :    { "hasAttribute" : "Version2" }
}
json.dump([json1, json2], f_out, indent=4, sort_keys=True)
f_out.write('''
</script>
</tr>
''')

if mode == 'r' or mode == 'j':
f_out.write('''      </table>
''')
elif mode == 't' or mode == 'u':
f_out.write("\n")

def write_header(f_out, mode):
if mode == 'j' or mode == 'r':
f_out.write('''<html>
<head>
</head>
<body vocab="http://www.w3.org/nw/prov#" prefix="vo: https://orion.tw.rpi.edu/~blee/VersionOntology.owl# v1: http://ngdb.com/v1/ v2: http://ngdb.com/v2/">
''')
if mode == 'j':
f_out.write('''  <script type="application/ld+json">
''')
json1 = {
"@context":context,
"@type":"vo:Version",
"@id":"Version1",
"label":"ngdbv1"
}
json2 = {
"@context":context,
"@type":"vo:Version",
"@id":"Version2",
"label":"DB_final-55-7262_2015_03_08.xlsx"
}
json.dump([json1,json2], f_out, indent=4, sort_keys=True)
f_out.write("\n  </script>\n")
if mode == 'u':
f_out.write("""@prefix vo: <http://orion.tw.rpi.edu/~blee/VersionOntology.owl#> .
@prefix skos: <http://www.w3.org/2004/02/skos/core#> .
@prefix rdf: <http://www.w3.org/1999/02/22-rdf-syntax-ns#> .
@prefix rdfs: <http://www.w3.org/2000/01/rdf-schema#> .
@prefix xml: <http://www.w3.org/XML/1998/namespace> .
@prefix xsd: <http://www.w3.org/2001/XMLSchema#> .

<http://example.com/NG/Version1> a vo:Version ;
skos:prefLabel "ngdbv1" .

<http://example.com/NG/Version2> a vo:Version ;
skos:prefLabel "DB_final-55-7262_2015_03_08.xlsx" .

""")

def write_footer(f_out, mode):
if mode == 'r':
f_out.write('</body>\n</html>')

def get_indicator_map(excel_files):
indicator_map = {}
for excel_file in excel_files:
print 'Importing: ' + excel_file
file_workbook = xlrd.open_workbook(excel_file)
file_sheet = file_workbook.sheet_by_index(0)
indicators = file_sheet.col(0)
for i in range(4, file_sheet.nrows):
indicator_map[indicators[i].value] = excel_file
return indicator_map

def compare(v1s, v2, fn_out, mode):
indicator_map = get_indicator_map(v1s)
i_keys = indicator_map.keys()
v2_workbook = xlrd.open_workbook(v2)
f_out = open(fn_out, 'w')

v2_sheet = v2_workbook.sheet_by_index(0)
v2_keys = [i.value for i in v2_sheet.col(0)]

converted = [index_convert(i) for i in range(0,54)]
new_col = [i for i in range(0, v2_sheet.ncols) if index_convert(i) == -1]
new_row = [(i, v2_sheet.cell(i,0).value) for i in range(3, v2_sheet.nrows) if v2_sheet.cell(i,0).value not in i_keys]
old_col = [i for i in range(0,194) if i not in converted]
old_row = [(i, indicator_map.get(i, None)) for i in i_keys if i not in v2_keys]

write_header(f_out, mode)
write_added(v2, new_col, new_row, f_out, mode)
write_removed(v2, old_col, old_row, f_out, mode)

if mode == 'r' or mode == 'j':
f_out.write('''
<h3>Change Log</h3>
''')
elif mode == 't':
f_out.write("Change Log\n")

workbook_name = ''
for i in range(3,v2_sheet.nrows):
v2_row = v2_sheet.row(i)
#workbook_name = v1_file
if v2_row[0].value in [j for i, j in new_row] or v2_row[0].value in [i for i, j in old_row]:
continue
if workbook_name == indicator_map.get(v2_row[0].value, None):
pass
else:
workbook_name = indicator_map.get(v2_row[0].value, None)
v1_workbook = xlrd.open_workbook(workbook_name)
v1_sheet = v1_workbook.sheet_by_index(0)
v1_col = v1_sheet.col(0)
v1_col = [j.value for j in v1_col]
#print v2_row[0].value
v1_index = v1_col.index(v2_row[0].value)
v1_row = v1_sheet.row(v1_index)
write_modify(v1_row, v2_row, workbook_name, f_out, mode)

write_footer(f_out, mode)
f_out.close()

if __name__ == "__main__":
if '-json' in sys.argv:
mode = 'j'
out_name = 'changelog_json.html'
elif '-rdfa' in sys.argv:
mode = 'r'
out_name = 'changelog_test.html'
elif '-txt' in sys.argv:
mode = 't'
out_name = 'changelog.txt'
elif '-ttl' in sys.argv:
mode = 'u'
out_name = 'changelog.ttl'

v2_dir = join(separator, 'data', 'NGdata', 'v2')
v1_dir = join(separator, 'data', 'NGdata', 'v1')

excel_files = glob.glob("/data/NGdata/v1/DB_HE_6733.xlsx") #join(v1_dir, '*.xlsx'))

v1_file = join(v1_dir, 'America_906.xlsx')
v2_file = join(v2_dir, 'DB_final-55-7262_2015_03_08.xlsx')

compare(excel_files, v2_file, out_name, mode)
\end{minted}

\chapter{NOBLE GAS CHANGE LOG GENERATOR VERSION 2 TO 3}

\begin{minted}[linenos, frame=lines, baselinestretch=1.2, breaklines]{Python}
from os.path import join, dirname, abspath, isfile
from os import sep as separator
import xlrd, sys, json
import glob
import re


def index_convert(index1):
return index1

def test_alignment():
for i in range(0, 54):
print 'version2: {:5} version1: {:5}'.format(i, index_convert(i))

def compare_print(mode, key, val1, val2, v1_file, v1_index = 0, v2_index = 0, changelog = None):
if changelog:
if mode == 'r':
out = u'''        <tr  about="Change{}{}" typeof="vo:ModifyChange">
<td align="right" rev="vo:Undergoes" resource="v2:Attribute{}{}v2" typeof="vo:Attribute">{:2}({})</td>
<td property="vo:resultsIn" resource="v3:Attribute{}{}v3" typeof="vo:Attribute">{:2}</td>
<td>{:>10}</td>
<td>{:>10}</td>
<span about="Version2" property="vo:hasAttribute" resource="v2:Attribute{}{}v2"></span>
<span about="Version3" property="vo:hasAttribute" resource="v3:Attribute{}{}v3"></span>
</tr>\n'''.format(key, v2_index, key, v1_index, v1_index, v1_file, key, v2_index, v2_index, val1, val2, key, v1_index, key, v2_index)
elif mode == 'j':
out = u'''        <tr  id="ModifyChange{}{}">
<td align="right">{:2}</td>
<td>{:2}</td>
<td>{:>10}</td>
<td>{:>10}</td>
<script type="application/ld+json">\n'''.format(key, v2_index, v1_index, v2_index, val1, val2)
elif mode == 't':
out = u"{:2}\t{:2}\t{:>10}\t{:>10}\n".format(v1_index, v2_index, val1, val2)
elif mode == 'u':
out = u"""<http://example.com/NG/Version2> vo:hasAttribute <http://example.com/NG/Version2/%s> ;
vo:hasAttribute <http://example.com/NG/Version2/Column%i> .
<http://example.com/NG/Version2/%s> a vo:Attribute ;
vo:undergoes <http://example.com/Changelog#ModifyChange%s%i> .
<http://example.com/NG/Version2/Column%i> a vo:Attribute ;
vo:undergoes <http://example.com/Changelog#ModifyChange%s%i> .
<http://example.com/Changelog#ModifyChange%s%i> a vo:ModifyChange ;
vo:resultsIn <http://example.com/NG/Version3/%s> ;
vo:resultsIn <http://example.com/NG/Version3/Column%i> .
<http://example.com/NG/Version3> vo:hasAttribute <http://example.com/NG/Version3/%s> ;
vo:hasAttribute <http://example.com/NG/Version3/Column%i> .

"""%(key, v1_index, key, key, v2_index, v1_index, key, v2_index, key, v2_index, key, v2_index, key, v2_index)
changelog.write(out.encode('utf8'))
if mode == 'j':
json1 = {
"@context":context,
"@type":"vo:Attribute" ,
"@id":"".join(["http://ngdb.com/v2/Attribute", key, str(v1_index)]) ,
"label":key ,
"undergoes":"".join([host, "ModifyChange", key, str(v2_index)]) ,
"@reverse" :    { "hasAttribute" : "Version2" }
}
json2 = {
"@context":context,
"@type":"vo:ModifyChange",
"@id":"".join([host, "ModifyChange", key, str(v2_index)]) ,
"resultsIn":"".join(["http://ngdb.com/v3/Attribute", key, str(v2_index)])
}
json3 = {
"@context":context,
"@type":"vo:Attribute" ,
"@id":"".join(["http://ngdb.com/v3/Attribute", key, str(v2_index)]) ,
"label":key ,
"@reverse" :    { "hasAttribute" : "Version3" }
}
json.dump([json1, json2, json3], changelog, indent=4, sort_keys=True)
changelog.write('''
</script>
</tr>
''')
else:
print '{:5}  version2: {:10} version3: {:10}'.format(key, val1, val2)

labels = {}

context = "https://orion.tw.rpi.edu/~blee/provdist/GCMD/VO.jsonld"
host = "http://orion.tw.rpi.edu/~blee/provdist/NobleGas/changelog_json.html#"
#test_alignment()


#print v2_row[0].value
#print indicator_map[v2_row[0].value]

#v1_workbook = xlrd.open_workbook(v1_file)
#v1_sheet = v1_workbook.sheet_by_index(0)
#v1_row = v1_sheet.row(4)

def write_modify(r1, r2, workbook, f_out, mode):
if mode == 'r':
out = u'''  <div about="Version2" rel="vo:hasAttribute">
<div resource="v3:%s" typeof="vo:Attribute">
<span style="font-weight:bold" property="http://www.w3.org/2000/01/rdf-schema#label">%s</span>
<table rel="vo:Undergoes">
'''%(r2[0].value, r2[0].value)
elif mode == 'j':
out = u'''
<div about="v3:%s">
<span style="font-weight:bold" property="http://www.w3.org/2000/01/rdf-schema#label">%s</span>
<table>
'''%(r2[0].value, r2[0].value)
elif mode == 't':
out = u"%s\n"%(r2[0].value)
elif mode == 'u':
out = u""

if mode == 'r' or mode == 'j':
out = out+'''        <tr>
<th>Column v2</th>
<th>Column v3</th>
<th>Version 2</th>
<th>Version 3</th>
</tr>\n'''
elif mode == 't':
out = out+"Column v2\tColumn v3\tVersion 2\tVersion 3\n"
f_out.write(out.encode('utf8'))
#print '# Searching...'
#print '# Comparing...'
for i in range(0,54):
if r2[i].value != r1[i].value:
#compare_print(j, v1_row[index_convert(j)].value, v2_row[j].value)
compare_print(mode, r2[0].value, r1[i].value, r2[i].value, workbook.split('/')[-1], i, i, f_out)
if mode == 'r' or mode == 'j':
f_out.write('  </table></div><br>\n')
elif mode == 't' or mode == 'u':
f_out.write("\n")

def write_removed(v2, col, row, f_out, mode):
if mode == 'r' or mode == 'j':
f_out.write('''
<h3>Columns invalidated by %s</h3>
<table about="Version2">
'''%(v2.split('/')[-1]))
elif mode == 't':
f_out.write("\nColumns invalidated by %s\n"%(v2.split('/')[-1]))

print "Removed Column"
for i in col:
v1_value = labels.get(i, "")
if mode == 'r':
out = u'''        <tr resource="InvlidateChange%i" rev="vo:invalidatedBy" typeof="vo:InvalidateChange">
<td resource="Attribute%i" rev="vo:Undergoes" typeof="vo:Attribute">%i</td>
<td about="Attribute%i" property="http://www.w3.org/2000/01/rdf-schema#label">%s</td>
<span about="Version1" property="vo:hasAttribute" resource="Attribute%i"/>
</tr>
'''%(i, i, i, i, v1_value, i)
elif mode == 'j':
out = u'''        <tr id="InvlidateChange%i" about="InvlidateChange%i">
<td>%i</td>
<td>%s</td>
<script type="application/ld+json">
'''%(i, i, i, v1_value)
elif mode == 't':
out = u"%i\t%s\n"%(i, v1_value)
elif mode == 'u':
out = u"""<http://example.com/NG/Version2> vo:hasAttribute <http://example.com/NG/Version2/Column%i> .
<http://example.com/NG/Version2/Column%i> vo:undergoes <http://example.com/Changelog#InvalidateChange%i> .
<http://example.com/Changelog#InvalidateChange%i> a vo:InvalidateChange ;
vo:invalidatedBy <http://example.com/NG/Version3> .

"""%(i, i, i, i)
f_out.write(out.encode('utf8'))
if mode == 'j':
json1 = {
"@context":context,
"@type":"vo:Attribute" ,
"@id":"".join(["http://ngdb.com/v2/Attribute", str(i)]) ,
"label": v1_value,
"undergoes":"".join([host, "InvalidateChange", str(i)]) ,
"@reverse" :    { "hasAttribute" : "Version2" }
}
json2 = {
"@context":context,
"@type":"vo:InvalidateChange" ,
"@id": "".join([host, "InvalidateChange", str(i)]) ,
"invalidatedBy"  :   "Version3"
}
json.dump([json1, json2], f_out, indent=4, sort_keys=True)
f_out.write('''
</script>
</tr>
''')


if mode == 'r' or mode == 'j':
f_out.write('''      </table>
<h3>Rows invalidated by %s</h3>
<table about="Version2">
'''%(v2.split('/')[-1]))
elif mode == 't':
f_out.write("\nRows invalidated by %s\n"%(v2.split('/')[-1]))
elif mode == 'u':
f_out.write("\n")

print "Removed Row"
for i, j in sorted(row):#i is row #, j is row id
if mode == 'r':
out = u'''        <tr resource="InvlidateChange%s" rev="vo:invalidatedBy" typeof="vo:InvalidateChange">
<td resource="Attribute%s" rev="vo:Undergoes" typeof="vo:Attribute">%i</td>
<td about="Attribute%s" property="http://www.w3.org/2000/01/rdf-schema#label">%s</td>
<span about="Version2" property="vo:hasAttribute" resource="Attribute%s"/>
</tr>
'''%(j, j, i, j, j, j)
elif mode == 'j':
out = u'''        <tr id="InvlidateChange%s" about="InvlidateChange%s">
<td>%i</td>
<td>%s</td>
<script type="application/ld+json">
'''%(j, j, i,  j)
elif mode == 't':
out = u"%i\t%s\n"%(i, j)
elif mode == 'u':
out = u"""<http://example.com/NG/Version2> vo:hasAttribute <http://example.com/NG/Version2/%s> .
<http://example.com/NG/Version2/%s> vo:undergoes <http://example.com/Changelog#InvalidateChange%s> .
<http://example.com/Changelog#InvalidateChange%s> a vo:InvalidateChange ;
vo:invalidatedBy <http://example.com/NG/Version2> .

"""%(j, j, j, j)
f_out.write(out.encode('utf8'))
if mode == 'j':
json1 = {
"@context":context,
"@type":"vo:Attribute" ,
"@id":"".join(["http://ngdb.com/v2/Attribute", j]) ,
"label": j,
"undergoes":"".join([host, "InvalidateChange", j]) ,
"@reverse" :    { "hasAttribute" : "Version2" }
}
json2 = {
"@context":context,
"@type":"vo:InvalidateChange" ,
"@id": "".join([host, "InvalidateChange", j]) ,
"invalidatedBy"  :   "Version3"
}
json.dump([json1, json2], f_out, indent=4, sort_keys=True)
f_out.write('''
</script>
</tr>
''')
if mode == 'r' or mode == 'j':
f_out.write('''      </table>

''')
elif mode == 't' or mode == 'u':
f_out.write("\n")


def write_added(v2, col, row, f_out, mode):
if mode == 'r' or mode == 'j':
f_out.write('''
<h3>Columns added by %s</h3>
<table about="Version2" rel="vo:absentFrom">
'''%(v2.split('/')[-1]))
elif mode == 't':
f_out.write("\nColumns added by %s\n\n"%(v2.split('/')[-1]))

print "Added Column"
for i in col:
print i#, v2_value
if mode == 'r':
f_out.write('''        <tr about="AddChange%i" typeof="vo:AddChange">
<td property="vo:resultsIn" resource="Attribute%i" typeof="vo:Attribute">%i</td>
<td about="Attribute%i" property="http://www.w3.org/2000/01/rdf-schema#label"></td>
<span about="Version3" property="vo:hasAttribute" resource="Attribute%i"/>
</tr>
'''%(i, i, i, i, i))
elif mode == 'j':
f_out.write('''        <tr id="AddChange%i" about="v2:Attribute%i">
<td>%i</td>
<td></td>
<script type="application/ld+json">
'''%(i, i, i))
json1 = {
"@context":context,
"@type":"vo:AddChange" ,
"@id": "".join([host, "AddChange", str(i)]) ,
"resultsIn" :   "".join([ "http://ngdb.com/v3/Attribute", str(i)]),
"@reverse"  :   { "absentFrom": "Version2" }
}
json2 = {
"@context":context,
"@type":"vo:Attribute" ,
"@id":"".join(["http://ngdb.com/v3/Attribute", str(i)]) ,
"label":"" ,
"@reverse" :    { "hasAttribute" : "Version3" }
}
json.dump([json1, json2], f_out, indent=4, sort_keys=True)
f_out.write('''
</script>
</tr>
''')
elif mode == 't':
f_out.write("%i\t\n"%(i))	
elif mode == 'u':
f_out.write("""<http://example.com/NG/Version2> vo:absentFrom <http://example.com/Changelog#AddChange%i> .
<http://example.com/Changelog#AddChange%i> a vo:AddChange ;
vo:resultsIn <http://example.com/NG/Version3/Column%s> .
<http://example.com/NG/Version3> vo:hasAttribute <http://example.com/NG/Version3/Column%s> .

"""%(i, i, i, i))
if mode == 'r' or mode == 'j':
f_out.write('''      </table>
<h3>Rows added by %s</h3>
<table about="Version2" rel="vo:absentFrom">
'''%(v2.split('/')[-1]))
elif mode == 't':
f_out.write("\nRows added by %s\n\n"%(v2.split('/')[-1]))
elif mode == 'u':
f_out.write("\n")

print "Added Row"
for i, j in row:#i is the row #, j is the id
if mode == 'r':	                #print i, v2_sheet.cell(i,0).value
out = u'''        <tr about="AddChange%s" typeof="vo:AddChange">
<td property="vo:resultsIn" resource="Attribute%s" typeof="vo:Attribute">%i</td>
<td about="Attribute%s" property="http://www.w3.org/2000/01/rdf-schema#label">%s</td>
<span about="Version3" property="vo:hasAttribute" resource="Attribute%s"/>
</tr>
'''%(j, j, i, j, j, j)
elif mode == 'j':
out = u'''        <tr id="AddChange%s" about="v3:Attribute%s">
<td>%i</td>
<td property="http://www.w3.org/2000/01/rdf-schema#label">%s</td>
<script type="application/ld+json">
'''%(j, j, i, j)
elif mode == 't':
out = u"%i\t%s\n"%(i, j)
elif mode == 'u':
out = u"""<http://example.com/NG/Version2> vo:absentFrom <http://example.com/Changelog#AddChange%s> .
<http://example.com/Changelog#AddChange%s> a vo:AddChange ;
vo:resultsIn <http://example.com/NG/Version3/%s> .
<http://example.com/NG/Version3> vo:hasAttribute <http://example.com/NG/Version3/%s> .

"""%(j, j, j, j)
f_out.write(out.encode('utf8'))
if mode == 'j':
json1 = {
"@context":context,
"@type":"vo:AddChange" ,
"@id": "".join([host, "AddChange", j]) ,
"resultsIn" :   "".join([ "http://ngdb.com/v3/Attribute", j]),
"@reverse"  :   { "absentFrom": "Version2" }
}
json2 = {
"@context":context,
"@type":"vo:Attribute" ,
"@id":"".join(["http://ngdb.com/v3/Attribute", j]) ,
"label": j ,
"@reverse" :    { "hasAttribute" : "Version2" }
}
json.dump([json1, json2], f_out, indent=4, sort_keys=True)
f_out.write('''
</script>
</tr>
''')

if mode == 'r' or mode == 'j':
f_out.write('''      </table>
''')
elif mode == 't' or mode == 'u':
f_out.write("\n")

def write_header(f_out, mode):
if mode == 'j' or mode == 'r':
f_out.write('''<html>
<head>
</head>
<body vocab="http://www.w3.org/nw/prov#" prefix="vo: https://orion.tw.rpi.edu/~blee/VersionOntology.owl# v2: http://ngdb.com/v2/ v3: http://ngdb.com/v3/">
''')
if mode == 'j':
f_out.write('''  <script type="application/ld+json">
''')
json1 = {
"@context":context,
"@type":"vo:Version",
"@id":"Version2",
"label":"DB_final-55-7262_2015_03_08.xlsx"
}
json2 = {
"@context":context,
"@type":"vo:Version",
"@id":"Version3",
"label":"NG_DB_final_2017_07_01.xlsx"
}
json.dump([json1,json2], f_out, indent=4, sort_keys=True)
f_out.write("\n  </script>\n")
if mode == 'u':
f_out.write("""@prefix vo: <http://orion.tw.rpi.edu/~blee/VersionOntology.owl#> .
@prefix skos: <http://www.w3.org/2004/02/skos/core#> .
@prefix rdf: <http://www.w3.org/1999/02/22-rdf-syntax-ns#> .
@prefix rdfs: <http://www.w3.org/2000/01/rdf-schema#> .
@prefix xml: <http://www.w3.org/XML/1998/namespace> .
@prefix xsd: <http://www.w3.org/2001/XMLSchema#> .

<http://example.com/NG/Version3> a vo:Version ;
skos:prefLabel "NG_DB_final_2017_07_01.xlsx" .

<http://example.com/NG/Version2> a vo:Version ;
skos:prefLabel "DB_final-55-7262_2015_03_08.xlsx" .

""")

def write_footer(f_out, mode):
if mode == 'r':
f_out.write('</body>\n</html>')

def get_indicator_map(excel_files):
indicator_map = {}
for excel_file in excel_files:
print 'Importing: ' + excel_file
file_workbook = xlrd.open_workbook(excel_file)
file_sheet = file_workbook.sheet_by_index(0)
indicators = file_sheet.col(0)
for i in range(4, file_sheet.nrows):
indicator_map[indicators[i].value] = excel_file
return indicator_map

def compare(v1s, v2, fn_out, mode):
v1_workbook = xlrd.open_workbook(v1s)
v1_sheet = v1_workbook.sheet_by_index(0)
i_keys = {j.value:i for i,j in enumerate(v1_sheet.col(0)[3:],3)}

v2_workbook = xlrd.open_workbook(v2)
v2_sheet = v2_workbook.sheet_by_index(0)
v2_keys = {j.value:i for i,j in enumerate(v2_sheet.col(0)[3:],3)}

f_out = open(fn_out, 'w')

new_col = [i for i in range(0, v2_sheet.ncols) if index_convert(i) == -1]
new_row = [(v2_keys[i], i) for i in v2_keys.keys() if i not in i_keys.keys()]
old_col = [i for i in range(0, v1_sheet.ncols) if index_convert(i) == -1]
old_row = [(i_keys[i], i) for i in i_keys.keys() if i not in v2_keys.keys()]

write_header(f_out, mode)
write_added(v2, new_col, new_row, f_out, mode)
write_removed(v2, old_col, old_row, f_out, mode)

if mode == 'r' or mode == 'j':
f_out.write('''
<h3>Change Log</h3>
''')
elif mode == 't':
f_out.write("Change Log\n")

workbook_name = ''
for i in range(3,v2_sheet.nrows):
v2_row = v2_sheet.row(i)
#workbook_name = v1_file
if v2_row[0].value in [j for i, j in new_row] or v2_row[0].value in [j for i, j in old_row]:
continue
v1_row = v1_sheet.row(i_keys[v2_row[0].value])
write_modify(v1_row, v2_row, workbook_name, f_out, mode)

write_footer(f_out, mode)
f_out.close()

if __name__ == "__main__":
if '-json' in sys.argv:
mode = 'j'
out_name = 'isotope2_3_json.html'
elif '-rdfa' in sys.argv:
mode = 'r'
out_name = 'isotope2_3_rdfa.html'
elif '-txt' in sys.argv:
mode = 't'
out_name = 'changelog2_3.txt'
elif '-ttl' in sys.argv:
mode = 'u'
out_name = 'changelog2_3.ttl'

v2_dir = join(separator, 'data', 'NGdata', 'v2')
v3_dir = join(separator, 'data', 'NGdata', 'v3')

v2_file = join(v2_dir, 'DB_final-55-7262_2015_03_08.xlsx')
v3_file = join(v3_dir, 'NG_DB_final_2017_07_01.xlsx')

compare(v2_file, v3_file, out_name, mode)
\end{minted}

\chapter{GLOBAL CHANGE MASTER DIRECTORY CHANGE LOG GENERATOR VERSION JUNE 12, 2012 TO VERSION 8.4.1}

\begin{minted}[linenos, frame=lines, baselinestretch=1.2, breaklines]{Python}
import glob, json, rdflib, re
from rdflib import URIRef, Literal, Namespace
from rdflib.namespace import RDF, SKOS

def GCMDChangeLogGenerator(GCMDfile):
GCMD = Namespace("http://gcmdservices.gsfc.nasa.gov/kms/concept/")

#GCMDfile = ['GCMD8_3.rdf', 'GCMD8_4.rdf','GCMD8_4_1.rdf']
numbers = [re.search('GCMD(.*).rdf', i).group(1).replace("_","") for i in GCMDfile]
print numbers
filename = 'ChangelogGCMD'+"_".join(numbers)+'.html'
output = open(filename, 'w') #'/home/blee/provdist/GCMD/webGCMD83_84.html', 'w')
#output = codecs.open('/home/blee/GCMD/GCMD8_3to8_4.html', mode='w', encoding='utf-8')

g0 = rdflib.Graph()
g0.parse(GCMDfile[0])
g1 = rdflib.Graph()
g1.parse(GCMDfile[1])

ver = [re.search('GCMD(.*).rdf', i).group(1).replace("_",".") for i in GCMDfile]#['8.3', '8.4']
print ver
new = g1-g0
old = g0-g1

#Get the date for the change notes in the new changes made in version 2
#This will help determine if some concepts were changed without being moved
#Their change notes should have a date on the same day as the new additions.
#This is probably a bad way of determining this.
date = g1.value(new.value(predicate=RDF.type, object=SKOS.Concept), SKOS.changeNote).split()[0]
context = "https://orion.tw.rpi.edu/~blee/provdist/GCMD/VO.jsonld"

##################
###   Header   ###
##################

output.write('''<html>
<head>
<link href="https://maxcdn.bootstrapcdn.com/bootstrap/3.3.7/css/bootstrap.min.css" rel="stylesheet" integrity="sha384-BVYiiSIFeK1dGmJRAkycuHAHRg32OmUcww7on3RYdg4Va+PmSTsz/K68vbdEjh4u" crossorigin="anonymous">
</head>
<body vocab="http://www.w3.org/nw/prov#" prefix="gcmd: http://gcmdservices.gsfc.nasa.gov/kms/concept/">
<h2 property="http://purl.org/dc/terms/title">
<span about="gcmd:concept_scheme/sciencekeywords/?format=xml&version=%s" property="http://www.w3.org/2000/01/rdf-schema#label">%s</span> to 
<span about="gcmd:concept_scheme/sciencekeywords/?format=xml&version=%s" property="http://www.w3.org/2000/01/rdf-schema#label">%s</span>
'''%(ver[0], GCMDfile[0], ver[1], GCMDfile[1]))

output.write('''      <script type="application/ld+json">
[
{
"@context" : "%s" ,
"type"	:	"vo:Version" ,
"id"	:	"gcmd:concept_scheme/sciencekeywords/?format=xml&version=%s" ,
"label" :       "%s"
},
{
"@context" : "%s",
"type"  :       "vo:Version" ,
"id"    :       "gcmd:concept_scheme/sciencekeywords/?format=xml&version=%s" ,
"label" :       "%s"
}
]
</script>
</h2>
'''%(context, ver[0], GCMDfile[0], context, ver[1], GCMDfile[1]))

#################
###   ADDED   ###
#################

output.write('''
<h3>Concepts added to %s</h3>
<table about="gcmd:concept_scheme/sciencekeywords/?format=xml&version=%s" class="table table-striped">
<tr>
<th>Link</th>
<th>Concept</th>
<th>Change Note</th>
</tr>
'''%(GCMDfile[1], ver[1]))

c = 0

for i in new.subjects(RDF.type, SKOS.Concept):
changeNote = "<br>\n              ".join(g1.objects(i, SKOS.changeNote))
output.write((u'''        <tr id="AddChange%i" about="%s?version=%s">
<td>
<a href="%s?version=%s">Link</a>
</td>
<td property="http://www.w3.org/2004/02/skos/core#prefLabel">%s</td>
<td property="http://www.w3.org/2004/02/skos/core#changeNote">%s</td>
'''%(c, str(i), ver[1], str(i), ver[1], g1.value(i, SKOS.prefLabel), changeNote)).encode('utf8'))
output.write((u'''
<script type="application/ld+json">
[
{
"@context" : "%s" ,
"type"	:	"vo:AddChange" ,
"id"	:	"this:AddChange%i" ,
"resultsIn" :	"gcmd:%s?version=%s" ,
"@reverse"  :	{ "absentFrom": "gcmd:concept_scheme/sciencekeywords/?format=xml&version=%s" }
},
{
"@context" : "%s" ,
"type"	:	"vo:Attribute" ,
"id"	:	"gcmd:%s?version=%s" ,
"label" :	"%s" ,
"@reverse" :	{ "hasAttribute" : "gcmd:concept_scheme/sciencekeywords/?format=xml&version=%s" }
}
]
</script>
</tr>
'''%(context, c, i.split('/')[-1], ver[1], ver[0], context, i.split('/')[-1], ver[1], g1.value(i, SKOS.prefLabel), ver[1])).encode('utf8'))
c += 1

output.write('''      </table>

''')

#print date

#################
### REMOVED   ###
#################

output.write('''
<h3>Concepts removed from %s</h3>
<table about="gcmd:concept_scheme/sciencekeywords/?format=xml&version=%s" class="table table-striped">
<tr>
<th>Link</th>
<th>Concept</th>
</tr>
'''%(GCMDfile[0], ver[0]))

c = 0

for i in old.subjects(RDF.type, SKOS.Concept):#Reverse relations due to ordering and structure
output.write((u'''        <tr id="InvlidateChange%i" about="%s?version=%s">
<td>
<a href="%s?version=%s">Link</a>
</td>
<td property="http://www.w3.org/2004/02/skos/core#prefLabel">%s</td>
'''%(c, str(i), ver[0], str(i), ver[0], g0.value(i, SKOS.prefLabel), )).encode('utf8'))
output.write((u'''          <script type="application/ld+json">
[
{
"@context" : "%s" ,
"type"	:	"vo:Attribute" ,
"id"	:	"gcmd:%s?version=%s" ,
"label"	:	"%s" ,
"undergoes" :	"this:InvalidateChange%i" ,
"@reverse" :	{ "hasAttribute" : "gcmd:concept_scheme/sciencekeywords/?format=xml&version=%s" }
},
{
"@context" : "%s" ,
"type"	:	"vo:InvalidateChange" ,
"id"	:	"this:InvalidateChange%i" ,
"invalidatedBy"	:	"gcmd:concept_scheme/sciencekeywords/?format=xml&version=%s"
}
]
</script>
</tr>
'''%(context, i.split('/')[-1], ver[0], g0.value(i, SKOS.prefLabel), c, ver[0], context, c, ver[1])).encode('utf8'))
c += 1
output.write('''      </table>

''')

##################
###   Modify   ###
##################

output.write('''
<h3>Moved Concepts</h3>
<table class="table table-striped">
<tr>
<th>Link v1</th>
<th>Link v2</th>
<th>Label</th>
<th>Old Parent</th>
<th>New Parent</th>
</tr>\n
''')

c = 0

for i in g1.subjects(RDF.type, SKOS.Concept):
if (i, None, None) in g0:
b0 = g0.value(i, SKOS.broader)
b1 = g1.value(i, SKOS.broader)
if b0 != b1:
output.write((u'''        <tr id="MoveChange%i" about="%s?version=%s">
<td><a href=%s?version=%s>Link</a></td>
<td><a href=%s?version=%s>Link</a></td>
<td property="http://www.w3.org/2004/02/skos/core#prefLabel">%s</td>
<td about="%s?version=%s" property="http://www.w3.org/2004/02/skos/core#prefLabel">%s</td>
<td about="%s?version=%s" property="http://www.w3.org/2004/02/skos/core#prefLabel">%s</td>
'''%(c, str(i), ver[1],
str(i), ver[0],
str(i), ver[1],
g1.value(i, SKOS.prefLabel),
b0, ver[0], g0.value(b0, SKOS.prefLabel),
b1, ver[1], g1.value(b1, SKOS.prefLabel))  ).encode('utf8'))

output.write((u'''          <script  type="application/ld+json">
[
{
"@context" : "%s" ,
"type"	:	"vo:Attribute" ,
"id"	:	"gcmd:%s?version=%s" ,
"label"	:	"%s" ,
"undergoes" :	"this:MoveChange%i" ,
"@reverse" :	{ "hasAttribute" : "gcmd:concept_scheme/sciencekeywords/?format=xml&version=%s" }
},
{
"@context" : "%s" ,
"type"	:	"vo:MoveChange" ,
"id"	:	"this:MoveChange%i" ,
"resultsIn"	:	"gcmd:%s?version=%s"
},
{
"@context" : "%s" ,
"type"	:	"vo:Attribute" ,
"id"	:	"gcmd:%s?version=%s" ,
"label"	:	"%s" ,
"@reverse" :	{ "hasAttribute" : "gcmd:concept_scheme/sciencekeywords/?format=xml&version=%s" }
}
]
</script>
</tr>
'''%(context, i.split('/')[-1], ver[0], g0.value(i, SKOS.prefLabel), c, ver[0], 
context, c, i.split('/')[-1], ver[1], 
context, i.split('/')[-1], ver[1], g1.value(i, SKOS.prefLabel), ver[1])  ).encode('utf8'))
c += 1

output.write('''      </table>

''')

##################################
###   NON-STRUCTURAL CHANGES   ###
##################################

output.write('''
<h3>Non-Structural Changes</h3>
<table class="table table-striped">
<tr>
<th>Link v1</th>
<th>Link v2</th>
<th>Label</th>
<th>Change Notes</th>
</tr>\n
''')

c = 0

for i in g1.subjects(RDF.type, SKOS.Concept):
if (i, None, None) in g0:
b0 = g0.value(i, SKOS.broader)
b1 = g1.value(i, SKOS.broader)
if b0 == b1:
new_note = False
notes = []
for note in g1.objects(i, SKOS.changeNote):
note_date = note.split()[0]
#print note_date
if note_date == date:
new_note = True
notes.append(note)
if new_note:
output.write((u'''        <tr id="ModifyChange%i" about="%s?version=%s">
<td><a href=%s?version=%s>Link</a></td>
<td><a href=%s?version=%s>Link</a></td>
<td property="http://www.w3.org/2004/02/skos/core#prefLabel">%s</td>
<td property="http://www.w3.org/2004/02/skos/core#changeNote">%s</td>
'''%(c, str(i), ver[1],
str(i), ver[0],
str(i), ver[1],
g1.value(i, SKOS.prefLabel),
"<br>\n              ".join(notes)
)).encode('utf8'))
output.write((u'''
</tr>
'''%()).encode('utf8'))
c += 1

output.write('''      </table>

''')
output.write("\t</body>\n</html>")
output.close()

if __name__ == "__main__":
GCMDfiles = sorted(glob.glob("*.rdf"))
for i in range(len(GCMDfiles)-1):
print "Starting",GCMDfiles[i-1],"and",GCMDfiles[i] #It's done this way because GCMDJun1220012 sorts to the last item
GCMDChangeLogGenerator([GCMDfiles[i-1],GCMDfiles[i]])
\end{minted}

\chapter{GLOBAL CHANGE MASTER DIRECTORY CHANGE LOG GENERATOR VERSION 8.4.1 TO 8.5}
\begin{minted}[linenos, frame=lines, baselinestretch=1.2, breaklines]{Python}
import glob, json, rdflib, re
from rdflib import URIRef, Literal, Namespace
from rdflib.namespace import RDF, SKOS

def GCMDChangeLogGenerator(GCMDfile):
GCMD = Namespace("http://gcmdservices.gsfc.nasa.gov/kms/concept/")
GCMD8_5 = Namespace("https://gcmdservices.gsfc.nasa.gov/kms/concept/")

#GCMDfile = ['GCMD8_3.rdf', 'GCMD8_4.rdf','GCMD8_4_1.rdf']
numbers = [re.search('GCMD(.*).rdf', i).group(1).replace("_","") for i in GCMDfile]
print numbers
filename = 'ChangelogGCMD2'+"_".join(numbers)+'.html'
output = open(filename, 'w') #'/home/blee/provdist/GCMD/webGCMD83_84.html', 'w')
#output = codecs.open('/home/blee/GCMD/GCMD8_3to8_4.html', mode='w', encoding='utf-8')

g0 = rdflib.Graph()
g0.parse(GCMDfile[0])
g1 = rdflib.Graph()
g1.parse(GCMDfile[1])

ver = [re.search('GCMD(.*).rdf', i).group(1).replace("_",".") for i in GCMDfile]#['8.3', '8.4']
print ver
new = rdflib.Graph()
for s, p, o in g1.triples((None, RDF.type, SKOS.Concept)):
if not (GCMD[s.split('/')[-1]], p, o) in g0:
new.add((s, p, o))
old = rdflib.Graph()
for s, p, o in g0.triples((None, RDF.type, SKOS.Concept)):
if not (GCMD8_5[s.split('/')[-1]], p, o) in g1:
old.add((s, p, o))

#Get the date for the change notes in the new changes made in version 2
#This will help determine if some concepts were changed without being moved
#Their change notes should have a date on the same day as the new additions.
#This is probably a bad way of determining this.
date = g1.value(new.value(predicate=RDF.type, object=SKOS.Concept), SKOS.changeNote).split()[0]
context = "https://orion.tw.rpi.edu/~blee/provdist/GCMD/VO.jsonld"

##################
###   Header   ###
##################

output.write('''<html>
<head>
<link href="https://maxcdn.bootstrapcdn.com/bootstrap/3.3.7/css/bootstrap.min.css" rel="stylesheet" integrity="sha384-BVYiiSIFeK1dGmJRAkycuHAHRg32OmUcww7on3RYdg4Va+PmSTsz/K68vbdEjh4u" crossorigin="anonymous">
</head>
<body vocab="http://www.w3.org/nw/prov#" prefix="gcmd: http://gcmdservices.gsfc.nasa.gov/kms/concept/">
<h2 property="http://purl.org/dc/terms/title">
<span about="gcmd:concept_scheme/sciencekeywords/?format=xml&version=%s" property="http://www.w3.org/2000/01/rdf-schema#label">%s</span> to 
<span about="gcmd:concept_scheme/sciencekeywords/?format=xml&version=%s" property="http://www.w3.org/2000/01/rdf-schema#label">%s</span>
'''%(ver[0], GCMDfile[0], ver[1], GCMDfile[1]))

output.write('''      <script type="application/ld+json">
[
{
"@context" : "%s" ,
"type"	:	"vo:Version" ,
"id"	:	"gcmd:concept_scheme/sciencekeywords/?format=xml&version=%s" ,
"label" :       "%s"
},
{
"@context" : "%s",
"type"  :       "vo:Version" ,
"id"    :       "gcmd:concept_scheme/sciencekeywords/?format=xml&version=%s" ,
"label" :       "%s"
}
]
</script>
</h2>
'''%(context, ver[0], GCMDfile[0], context, ver[1], GCMDfile[1]))

#################
###   ADDED   ###
#################

output.write('''
<h3>Concepts added to %s</h3>
<table about="gcmd:concept_scheme/sciencekeywords/?format=xml&version=%s" class="table table-striped">
<tr>
<th>Link</th>
<th>Concept</th>
<th>Change Note</th>
</tr>
'''%(GCMDfile[1], ver[1]))

c = 0

for i in new.subjects(RDF.type, SKOS.Concept):
changeNote = "<br>\n              ".join(g1.objects(i, SKOS.changeNote))
output.write((u'''        <tr id="AddChange%i" about="%s?version=%s">
<td>
<a href="%s?version=%s">Link</a>
</td>
<td property="http://www.w3.org/2004/02/skos/core#prefLabel">%s</td>
<td property="http://www.w3.org/2004/02/skos/core#changeNote">%s</td>
'''%(c, str(i), ver[1], str(i), ver[1], g1.value(i, SKOS.prefLabel), changeNote)).encode('utf8'))
output.write((u'''
<script type="application/ld+json">
[
{
"@context" : "%s" ,
"type"	:	"vo:AddChange" ,
"id"	:	"this:AddChange%i" ,
"resultsIn" :	"gcmd:%s?version=%s" ,
"@reverse"  :	{ "absentFrom": "gcmd:concept_scheme/sciencekeywords/?format=xml&version=%s" }
},
{
"@context" : "%s" ,
"type"	:	"vo:Attribute" ,
"id"	:	"gcmd:%s?version=%s" ,
"label" :	"%s" ,
"@reverse" :	{ "hasAttribute" : "gcmd:concept_scheme/sciencekeywords/?format=xml&version=%s" }
}
]
</script>
</tr>
'''%(context, c, i.split('/')[-1], ver[1], ver[0], context, i.split('/')[-1], ver[1], g1.value(i, SKOS.prefLabel), ver[1])).encode('utf8'))
c += 1

output.write('''      </table>

''')

#print date

#################
### REMOVED   ###
#################

output.write('''
<h3>Concepts removed from %s</h3>
<table about="gcmd:concept_scheme/sciencekeywords/?format=xml&version=%s" class="table table-striped">
<tr>
<th>Link</th>
<th>Concept</th>
</tr>
'''%(GCMDfile[0], ver[0]))

c = 0

for i in old.subjects(RDF.type, SKOS.Concept):#Reverse relations due to ordering and structure
output.write((u'''        <tr id="InvlidateChange%i" about="%s?version=%s">
<td>
<a href="%s?version=%s">Link</a>
</td>
<td property="http://www.w3.org/2004/02/skos/core#prefLabel">%s</td>
'''%(c, str(i), ver[0], str(i), ver[0], g0.value(i, SKOS.prefLabel), )).encode('utf8'))
output.write((u'''          <script type="application/ld+json">
[
{
"@context" : "%s" ,
"type"	:	"vo:Attribute" ,
"id"	:	"gcmd:%s?version=%s" ,
"label"	:	"%s" ,
"undergoes" :	"this:InvalidateChange%i" ,
"@reverse" :	{ "hasAttribute" : "gcmd:concept_scheme/sciencekeywords/?format=xml&version=%s" }
},
{
"@context" : "%s" ,
"type"	:	"vo:InvalidateChange" ,
"id"	:	"this:InvalidateChange%i" ,
"invalidatedBy"	:	"gcmd:concept_scheme/sciencekeywords/?format=xml&version=%s"
}
]
</script>
</tr>
'''%(context, i.split('/')[-1], ver[0], g0.value(i, SKOS.prefLabel), c, ver[0], context, c, ver[1])).encode('utf8'))
c += 1
output.write('''      </table>

''')

##################
###   Modify   ###
##################

output.write('''
<h3>Moved Concepts</h3>
<table class="table table-striped">
<tr>
<th>Link v1</th>
<th>Link v2</th>
<th>Label</th>
<th>Old Parent</th>
<th>New Parent</th>
</tr>\n
''')

c = 0

for i in g1.subjects(RDF.type, SKOS.Concept):
i_ = GCMD[i.split('/')[-1]]
if (i_, None, None) in g0:
b0 = g0.value(i_, SKOS.broader)
b1 = g1.value(i, SKOS.broader)
if b1 != None:
b1_ = GCMD[b1.split('/')[-1]]
if b0 != b1_:
output.write((u'''        <tr id="MoveChange%i" about="%s?version=%s">
<td><a href=%s?version=%s>Link</a></td>
<td><a href=%s?version=%s>Link</a></td>
<td property="http://www.w3.org/2004/02/skos/core#prefLabel">%s</td>
<td about="%s?version=%s" property="http://www.w3.org/2004/02/skos/core#prefLabel">%s</td>
<td about="%s?version=%s" property="http://www.w3.org/2004/02/skos/core#prefLabel">%s</td>
'''%(c, str(i), ver[1],
str(i), ver[0],
str(i), ver[1],
g1.value(i, SKOS.prefLabel),
b0, ver[0], g0.value(b0, SKOS.prefLabel),
b1, ver[1], g1.value(b1, SKOS.prefLabel))  ).encode('utf8'))

output.write((u'''          <script  type="application/ld+json">
[
{
"@context" : "%s" ,
"type"	:	"vo:Attribute" ,
"id"	:	"gcmd:%s?version=%s" ,
"label"	:	"%s" ,
"undergoes" :	"this:MoveChange%i" ,
"@reverse" :	{ "hasAttribute" : "gcmd:concept_scheme/sciencekeywords/?format=xml&version=%s" }
},
{
"@context" : "%s" ,
"type"	:	"vo:MoveChange" ,
"id"	:	"this:MoveChange%i" ,
"resultsIn"	:	"gcmd:%s?version=%s"
},
{
"@context" : "%s" ,
"type"	:	"vo:Attribute" ,
"id"	:	"gcmd:%s?version=%s" ,
"label"	:	"%s" ,
"@reverse" :	{ "hasAttribute" : "gcmd:concept_scheme/sciencekeywords/?format=xml&version=%s" }
}
]
</script>
</tr>
'''%(context, i.split('/')[-1], ver[0], g0.value(i_, SKOS.prefLabel), c, ver[0], 
context, c, i.split('/')[-1], ver[1], 
context, i.split('/')[-1], ver[1], g1.value(i, SKOS.prefLabel), ver[1])  ).encode('utf8'))
c += 1

output.write('''      </table>

''')

output.write('''
<h3>Modified Concepts</h3>
<table class="table table-striped">
<tr>
<th>Link v1</th>
<th>Link v2</th>
<th>Label</th>
</tr>\n
''')

c = 0


##################################
###   NON-STRUCTURAL CHANGES   ###
##################################

output.write('''
<h3>Non-Structural Changes</h3>
<table class="table table-striped">
<tr>
<th>Link v1</th>
<th>Link v2</th>
<th>Label</th>
<th>Change Notes</th>
</tr>\n
''')

c = 0

for i in g1.subjects(RDF.type, SKOS.Concept):
i_ = GCMD[i.split('/')[-1]]
if (i_, None, None) in g0:
b0 = g0.value(i_, SKOS.broader)
b1 = g1.value(i, SKOS.broader)
if b1 != None:
b1_ = GCMD[b1.split('/')[-1]]
if b0 == b1_ and i != i_:
output.write((u'''        <tr id="NameChange%i" about="%s?version=%s">
<td><a href=%s?version=%s>Link</a></td>
<td><a href=%s?version=%s>Link</a></td>
<td property="http://www.w3.org/2004/02/skos/core#prefLabel">%s</td>
'''%(c, str(i), ver[1],
str(i_), ver[0],
str(i), ver[1],
g1.value(i, SKOS.prefLabel)
)))
output.write((u'''          <script  type="application/ld+json">
[
{
"@context" : "%s" ,
"type"	:	"vo:Attribute" ,
"id"	:	"%s?version=%s" ,
"label"	:	"%s" ,
"undergoes" :	"this:NameChange%i" ,
"@reverse" :	{ "hasAttribute" : "gcmd:concept_scheme/sciencekeywords/?format=xml&version=%s" }
},
{
"@context" : "%s" ,
"type"	:	"vo:ModifyChange" ,
"id"	:	"this:NameChange%i" ,
"resultsIn"	:	"%s?version=%s"
},
{
"@context" : "%s" ,
"type"	:	"vo:Attribute" ,
"id"	:	"%s?version=%s" ,
"label"	:	"%s" ,
"@reverse" :	{ "hasAttribute" : "gcmd:concept_scheme/sciencekeywords/?format=xml&version=%s" }
}
]
</script>
</tr>
'''%(context, i_, ver[0], g0.value(i_, SKOS.prefLabel), c, ver[0], 
context, c, i, ver[1], 
context, i, ver[1], g1.value(i, SKOS.prefLabel), ver[1])  ).encode('utf8'))
c += 1

output.write('''      </table>

''')

for i in g1.subjects(RDF.type, SKOS.Concept):
i_ = GCMD[i.split('/')[-1]]
if (i_, None, None) in g0:
b0 = g0.value(i_, SKOS.broader)
b1 = g1.value(i, SKOS.broader)
if b1 != None:
b1_ = GCMD[b1.split('/')[-1]]
if b0 == b1_:
new_note = False
notes = []
for note in g1.objects(i, SKOS.changeNote):
note_date = note.split()[0]
#print note_date
if note_date == date:
new_note = True
notes.append(note)
if new_note:
output.write((u'''        <tr id="ModifyChange%i" about="%s?version=%s">
<td><a href=%s?version=%s>Link</a></td>
<td><a href=%s?version=%s>Link</a></td>
<td property="http://www.w3.org/2004/02/skos/core#prefLabel">%s</td>
<td property="http://www.w3.org/2004/02/skos/core#changeNote">%s</td>
'''%(c, str(i), ver[1],
str(i), ver[0],
str(i), ver[1],
g1.value(i, SKOS.prefLabel),
"<br>\n              ".join(notes)
)).encode('utf8'))
output.write((u'''
</tr>
'''%()).encode('utf8'))
c += 1

output.write('''      </table>

''')
output.write("\t</body>\n</html>")
output.close()

if __name__ == "__main__":
GCMDfiles = sorted(glob.glob("*.rdf"))
GCMDfiles = ["GCMD8_5.rdf", "GCMD8_4_1.rdf"]
for i in range(len(GCMDfiles)-1):
print "Starting",GCMDfiles[i-1],"and",GCMDfiles[i] #It's done this way because GCMDJun1220012 sorts to the last item
GCMDChangeLogGenerator([GCMDfiles[i-1],GCMDfiles[i]])
\end{minted}

\chapter{TURTLE EXTRACTOR}
\begin{minted}[linenos, frame=lines, baselinestretch=1.2, breaklines]{Python}
from bs4 import BeautifulSoup
import glob, rdflib, json, re

def extracting(f, d):
#f = 'ChangelogGCMD70_80.html'
#d = 'Graph'+re.search('ChangelogGCMD(.*).html', f).group(1)+'.ttl'

fp = open(f)
soup = BeautifulSoup(fp, 'html5lib')
fp.close()

print 'extracting...'
js = soup.find_all('script')
items = [item for sublist in js for item in json.loads(sublist.text)]

print 'loading...'
g = rdflib.Graph()
g.parse(data = json.dumps(items), format='json-ld')

print 'writing...'
g.serialize(destination=d, format='turtle')
print 'written'

if __name__ == "__main__":
l = glob.glob('Changelog*.html')
for i in l:
d = 'Graph'+re.search('ChangelogGCMD(.*).html', i).group(1)+'.ttl'
print "Extracting: "+i
print "\tto", d
extracting(i, d)
\end{minted}

\chapter{MARINE BIODIVERSITY VIRTUAL LABORATORY CLASSIFIER COMPARISON}
\begin{minted}[linenos, frame=lines, baselinestretch=1.2, breaklines]{Python}
import urllib

class entry:
query = None
dist = None
freq = None
tax = None
poss = None

def __lt__(self, other):
if self.query == other.query:
return self.freq < other.freq
else:
return self.query < other.query

def is_number(s):
try:
float(s)
return True
except ValueError:
return False

def file_parse(f_name):
f = open(f_name, 'r')
found = {}
for i in mock:
found[i] = [[],[],0]
ambiguous = []
fp = []
fn_count = 0
fn = []
results = {}
total = 0
if f_name.split('_')[1] == 'spingo':
if f_name == "silva_spingo":
family[4] = 'Clostridiaceae_1'
else:
family[2] = 'Bacillaceae 1'
family[4] = 'Clostridiaceae 1'
for line in f:
x = line.split('\t')
entry_id = x[0].split('|')[0]
results[entry_id] = entry()
results[entry_id].query = x[0]
results[entry_id].freq = int(results[entry_id].query.split('|')[-1].split(':')[1])
results[entry_id].dist = float(x[1])
results[entry_id].tax = x[2:]

total += results[entry_id].freq

if not is_number(results[entry_id].tax[-1]):
results[entry_id].poss = results[entry_id].tax[-1].split(',')
results[entry_id].tax = results[entry_id].tax[:-1]
if results[entry_id].tax[-2] != 'AMBIGUOUS':
print results[entry_id].tax[-2]

clean_tax = [x for x in results[entry_id].tax if not is_number(x)]
if len(clean_tax) < 7:
clean_tax[0:0] = ['Pass', 'Pass', 'Pass', 'Pass']
del clean_tax[5]
results[entry_id].tax = clean_tax

if clean_tax[4] == 'AMBIGUOUS':
fn_count += 1
ambiguous.append(entry_id)
elif not clean_tax[4] in family:
fp.append((entry_id, 'family'))
elif clean_tax[5] == 'AMBIGUOUS':
for i in range(len(family)):
if clean_tax[4] == family[i]:
found[mock[i]][0].append(entry_id)
elif not clean_tax[5] in genus:
fp.append((entry_id, 'genus'))
elif clean_tax[6] == 'AMBIGUOUS':
for i in range(len(genus)):
if clean_tax[5] == genus[i]:
found[mock[i]][1].append(entry_id)
elif not ' '.join(clean_tax[6].split('_')) in mock:
fp.append((entry_id, 'species'))
else:
found[' '.join(clean_tax[6].split('_'))][2] += 1

elif f_name.split('_')[1] == 'gast':
skip = False
if f_name == "rdp_gast":
family[2] = 'Bacillaceae 1'
family[4] = 'Clostridiaceae 1'
for line in f:
if not skip:
skip = True
continue
x = line.split('\t')
entry_id = x[0].split('|')[0]
results[entry_id] = entry()
results[entry_id].query = x[0]
results[entry_id].freq = int(results[entry_id].query.split('|')[-1].split(':')[1])
results[entry_id].dist = float(x[2])
results[entry_id].tax = x[1].split(';')

total += results[entry_id].freq
temp_tax = results[entry_id].tax

if len(temp_tax) < 5:
fn_count += 1
ambiguous.append(entry_id)
elif not temp_tax[4] in family:
fp.append((entry_id, 'family'))
elif len(temp_tax) < 6:
for i in range(len(family)):
if temp_tax[4] == family[i]:
found[mock[i]][0].append(entry_id)
elif not temp_tax[5] in genus:
fp.append((entry_id, 'genus'))
elif len(temp_tax) < 7:
for i in range(len(genus)):
if temp_tax[5] == genus[i]:
found[mock[i]][1].append(entry_id)
elif not ' '.join(results[entry_id].tax[5:7]) in mock:
fp.append((entry_id, 'species'))
else:
found[' '.join(results[entry_id].tax[5:7])][2] += 1

fp_file = open("fp_"+f_name+".txt", 'w')
for fid, frank in fp:
fp_file.write("\t".join([fid, str(results[fid].freq), frank, str(results[fid].tax)])+"\n")
fp_file.close()

for i in found.keys():
if sum([len(found[i][0]), len(found[i][1]), found[i][2]]) == 0:
fn.append(i)
print fn
ttl_output(f_name, fp, fn, found, results)
total = len(results.keys())
print "\t".join(["False Positives:", str(len(fp))])
print "\t".join(["Ambiguous:", str(len(ambiguous))])
print "\t".join(["Total:", str(len(results.keys()))])
coverage = sum([len(found[i][0])+len(found[i][1])+found[i][2] for i in found.keys()])
print "\t".join(["Coverage:", str(coverage)])
print "\t".join(["Percentage:", str(float(coverage)/total*100)])
for k in sorted(mock):
print "\t".join([k+":", str(len(found[k][0])), str(len(found[k][1])), str(found[k][2])])

return results

def ttl_output(f_name, false_positive, false_negative, found, results):
fp_file = open("fp_"+f_name+".ttl", 'w')
fp_file.write("""@prefix vo: <http://orion.tw.rpi.edu/~blee/VersionOntology.owl#> .
@prefix mbvl: <http://example.com/MBVL/> .
@prefix next: <http://example.com/MBVL/%s> .
@prefix skos: <http://www.w3.org/2004/02/skos/core#> .
@prefix rdf: <http://www.w3.org/1999/02/22-rdf-syntax-ns#> .
@prefix rdfs: <http://www.w3.org/2000/01/rdf-schema#> .
@prefix xml: <http://www.w3.org/XML/1998/namespace> .
@prefix xsd: <http://www.w3.org/2001/XMLSchema#> .

<http://example.com/MBVL/db> a vo:Version .
<http://example.com/MBVL/%s> a vo:Version .\n"""%(f_name, f_name))
for i in range(len(false_positive)):
fid, j = false_positive[i]
clean_fid = urllib.quote(fid.split()[0])
if results[fid].freq > 10:
fp_file.write("""mbvl:db vo:absentFrom mbvl:AddChange%i .
next:%s a vo:Attribute .
mbvl:%s vo:hasAttribute next:%s .
mbvl:AddChange%i a vo:AddChange ;
vo:resultsIn next:%s .\n
"""%(i, clean_fid, f_name, clean_fid, i, clean_fid))
fp_file.close()

def get_output(tax1, alg1, tax2, alg2):
if alg1 == 'silva_spingo':
t1 = [x for x in tax1 if not is_number(x)]
del t1[5]
elif alg1 == 'rdp_spingo':
t1 = [x for x in tax1 if not is_number(x)]
print t1
del t1[1]
t1[0:0] = ['Pass', 'Pass', 'Pass', 'Pass']
else:
t1 = tax1

if alg2 == 'silva_spingo':
t2 = [x for x in tax2 if not is_number(x)]
del t2[5]
elif alg2 == 'rdp_spingo':
t2 = [x for x in tax2 if not is_number(x)]
del t2[1]
t2[0:0] = ['Pass', 'Pass', 'Pass', 'Pass']
else:
t2 = tax2

if alg1.split('_')[1] != alg2.split('_')[1]:
for j in range(7):
if t1[j] == 'Pass':
continue
if j >= len(t2):
if t1[j] != 'AMBIGUOUS':
return "\t".join([ '---', rank[j], str(t1[j:])])
elif j == 6:
if t1[j] == 'AMBIGUOUS':
return "\t".join(['+++', rank[j], str(t2[j:])])
elif t1[j] != '_'.join(t2[5:7]):
return "\t".join([ '>>>', rank[j], t1[j], str(tax2[5:])])
elif t1[j] == 'AMBIGUOUS':
return "\t".join([ '+++', rank[j], str(t2[j:])])
elif t1[j] != t2[j]:
return "\t".join([ '>>>', rank[j], str(t1[j]), str(t2[j:])])
return None
elif alg1.split('_')[1] == 'spingo':
for j in range(7):
if t1[j] == 'Pass' or t2[j] == 'Pass':
continue
if t1[j] == 'AMBIGUOUS':
if t2[j] == 'AMBIGUOUS':
return None
else:
return "\t".join(['+++', rank[j], str(t2[j:])])
elif t2[j] == 'AMBIGUOUS':
return "\t".join(['---', rank[j], str(t1[j:])])
elif t1[j] != t2[j]:
return "\t".join(['>>>', rank[j], str(t1[j]), str(t2[j])])
return None
elif alg1.split('_')[1] == 'gast':
if len(t1) > len(t2):
return "\t".join(['---', rank[len(t2)], str(t1[len(t2):])])
elif len(t1) < len(t2):
return "\t".join(['+++', rank[len(t1)], str(t2[len(t1):])])
else:
for j in range(len(t1)):
if t1[j] != t2[j]:
return "\t".join(['>>>', rank[j], str(t1[j]), str(t2[j])])
return None
else:
return -1

mock = ['Acinetobacter baumannii',
'Actinomyces odontolyticus',
'Bacillus cereus',
'Bacteroides vulgatus',
'Clostridium beijerinckii',
'Deinococcus radiodurans',
'Enterococcus faecalis',
'Escherichia coli',
'Helicobacter pylori',
'Lactobacillus gasseri',
'Listeria monocytogenes',
'Neisseria meningitidis',
'Porphyromonas gingivalis',
'Propionibacterium acnes',
'Pseudomonas aeruginosa',
'Rhodobacter sphaeroides',
'Staphylococcus aureus',
'Staphylococcus epidermidis',
'Streptococcus agalactiae',
'Streptococcus mutans',
'Streptococcus pneumoniae']

genus = [x.split()[0] for x in mock]

family = ['Moraxellaceae',
'Actinomycetaceae',
'Bacillaceae',
'Bacteroidaceae',
'Clostridiaceae',
'Deinococcaceae',
'Enterococcaceae',
'Enterobacteriaceae',
'Helicobacteraceae',
'Lactobacillaceae',
'Listeriaceae',
'Neisseriaceae',
'Porphyromonadaceae',
'Propionibacteriaceae',
'Pseudomonadaceae',
'Rhodobacteraceae',
'Staphylococcaceae',
'Staphylococcaceae',
'Streptococcaceae',
'Streptococcaceae',
'Streptococcaceae']

rank = ['Kingdom', 'Phylum', 'Class', 'Order', 'Family', 'Genus', 'Species', 'Strain']

if __name__ == "__main__":
fList = ['silva_spingo', 'silva_gast', 'rdp_spingo', 'rdp_gast']
#f1_name = fList[1]
#f2_name = fList[3]
#f1 = file_parse(f1_name)
#f2 = file_parse(f2_name)

for i in fList:
f1 = file_parse(i)
#out = file('silva_rdp_gast.txt', 'w')
#for i in sorted(f1.keys()):
#	x = get_output(f1[i].tax, f1_name, f2[i].tax, f2_name)
#	if not x == None:
#		out.write("\t".join([i,x])+"\n")
#out.close()
\end{minted}

\chapter{COLD LAND PROCESS FIELD EXPERIMENT CHANGE LOG GENERATOR}
\begin{minted}[linenos, frame=lines, baselinestretch=1.2, breaklines]{Python}
from glob import glob
from datetime import datetime
import csv
import dbfread
import sys

class Version:
unique_id = None

def __eq__(self, other):
if isinstance(other, self.__class__):
return self.unique_id == other.unique_id
else:
return self.unique_id == other

class StratVer(Version):

def __init__(self, headers, values):
v = [i.strip() for i in csv.reader([values]).next()]
self.data = {}
self.load_strat(headers, v)
self.unique_id = (self.data['PIT_NAME'], self.data['IOP'], int(self.data['TOP']), int(self.data['BOT']))		

def load_strat(self, headers, v):
for i in range(len(headers)):
if i == 2:
self.data[headers[i]] = datetime.strptime(v[i], "%Y-%m-%d").date()
elif i > 2 and i < 14:
if '.' in v[i]:
self.data[headers[i]] = float(v[i])
else:
self.data[headers[i]] = int(v[i])
else:
self.data[headers[i]] = v[i]

class SummVer(Version):

def __init__(self, headers, values):
v = [i.strip() for i in csv.reader([values]).next()]
self.data = {}
self.load_summ(headers, v)
self.unique_id = (self.data['PIT'], self.data['IOP'])

def load_summ(self, headers, v):
for i in range(len(headers)):
if i == 3:
try:
self.data[headers[i]] = datetime.strptime(v[i], "%Y-%m-%d").date()
except ValueError:
self.data[headers[i]] = None
elif (i > 3 and i < 7) or (i > 8 and i < 11) or (i > 11 and i < 34):
try:
self.data[headers[i]] = int(v[i])
except ValueError:
pass
try:
self.data[headers[i]] = float(v[i])
except ValueError:
self.data[headers[i]] = v[i]
else:
self.data[headers[i]] = v[i]

def helper_factory( section, mode, arg1=None):
if mode == "strat":
if section == "names":
return ("/data/ice/shape_files/pit_iop_v2_strat.dbf", "strat")
elif section == "key":
return "Key = (PIT, IOP, TOP, BOT)\n\n"
elif section == "mapper":
return strat_col_map
elif section == "table_ids":
return [(r['PIT'], r['IOP'], r['TOP'], r['BOT']) for r in arg1]
elif mode == "summary":
if section == "names":
return ("/data/ice/shape_files/pit_iop_v2_summary.dbf", "summary")
elif section == "key":
return "Key = (PIT, IOP)\n\n"
elif section == "mapper":
return summ_col_map
elif section == "table_ids":
return [(r['PIT'], r['IOP']) for r in arg1]

def strat_col_map(index, mode):
if mode == "ascii":
if index == "DATE_":
return "DATE"
elif index == "PIT":
return "PIT_NAME"
elif mode == "shape":
if index == "DATE":
return "DATE_"
elif index == "PIT_NAME":
return "PIT"
return index

def summ_col_map(index, mode):
if mode == "ascii" and index == "DATE_":
return "DATE"
elif mode == "shape" and index == "DATE":
return "DATE_"
return index

def ice_import(category):
if category == "strat":
fn = glob("/data/ice/ascii/*strat.csv")
elif category == "summary":
fn = glob("/data/ice/ascii/*summary.csv")

entries = {}
for i in fn:
f = open(i)
ll = f.readlines()
headers = [c.strip() for c in csv.reader([ll[0]]).next()]
for j in ll[2:]:
if category == "strat":
sv = StratVer(headers, j)
elif category == "summary":
sv = SummVer(headers, j)
entries[sv.unique_id] = sv
f.close()
return headers, entries

def generate_logs(table_name, text_name, f):
table = dbfread.DBF(table_name)
table_ids = helper_factory("table_ids", text_name, table)

headers, text_entries = ice_import(text_name)

f.write("Shape Entries not in Ascii Files\n")
for i in table_ids:
if i not in text_entries.keys():
f.write(str(i)+"\n")
f.write("\n")

f.write("Shape Columns not in Ascii Files\n")
for i in table.field_names:
if i not in [column_map(j, "shape") for j in headers]:
f.write(i+"\n")
f.write("\n")

f.write("Ascii Entries not in Shape Files\n")
for i in sorted(text_entries.keys()):
if i not in table_ids:
f.write(str(i)+"\n")
f.write("\n")

f.write("Ascii Columns not in Shape Files\n")
for i in headers:
if i not in [column_map(j, "ascii") for j in table.field_names]:
f.write(i+"\n")
f.write("\n")

c = 0
f.write("Modified Entries\n")
for r in table:
if text_name == "strat":
i = (r['PIT'], r['IOP'], r['TOP'], r['BOT'])
elif text_name == "summary":
i = (r['PIT'], r['IOP'])

if i in text_entries.keys():
v = text_entries[i]
start = True
for j in table.field_names:
if column_map(j, "ascii") not in v.data.keys():
pass
elif r[j] == 'NoData' and v.data[column_map(j, "ascii")] == '-999':
pass
elif r[j] != v.data[column_map(j, "ascii")]:
if start:
f.write("%s\n"%(str(v.unique_id)))
f.write("Column|\tShape|\tAscii\n")
start = False
f.write("%s|\t%s|\t%s\n"%(j, r[j], v.data[column_map(j, "ascii")]))
#print v.unique_id, j, r[j], v.data[attribute_map(j)]
c += 1
if not start:
f.write("\n")

if __name__ == "__main__":
if "-strat" in sys.argv:
table_name, text_name = helper_factory("names", "strat")
elif "-summ" in sys.argv:
table_name, text_name = helper_factory("names", "summary")

change_log = "changelog_"+text_name+".txt"
f = open(change_log, 'w')
f.write("Change Log %s to ascii files\n"%(table_name.split('/')[-1]))
f.write(helper_factory("key", text_name))

column_map = helper_factory("mapper", text_name)
generate_logs(table_name, text_name, f)
f.close()
\end{minted}

\section{Versions as a Class}

In the CLPX data, versions were encapsulated into individual classes so that computations between versions could be standardized and automated.
The code could handle ingesting a variety of different formats of data while keeping the core comparison functions independent of individualized fields.

\chapter{EARTH OBSERVING LABORATORY ANALYSER}
\begin{minted}[linenos, frame=lines, baselinestretch=1.2, breaklines]{Python}
import csv
from datetime import datetime

class dataset:
def __init__(self, d_id, d_title):
self.versions = {}
self.num = int(d_id)
self.title = d_title

def add_file(self, v_num, v_pub, v_crt, v_mod, f_name, f_crt, f_rev, f_notes):
if v_num not in self.versions:
self.versions[v_num] = eol_ver(v_num, v_pub, v_crt, v_mod)
self.versions[v_num].add_file(f_name, f_crt, f_rev, f_notes)

def __repr__(self):
out = ["%i: %s"%(self.num, self.title)]
for i in sorted(self.versions.keys()):
out.append(self.versions[i].string(4))
return "\n".join(out)

class eol_ver:
def __init__(self, v_num, t1, t2, t3):
self.num = v_num
self.files = []
self.v_pub = datetime.strptime(t1, "%Y-%m-%d %H:%M:%S")
self.v_crt = datetime.strptime(t2, "%Y-%m-%d %H:%M:%S")
self.v_mod = datetime.strptime(t3, "%Y-%m-%d %H:%M:%S")

def add_file(self, f_name, t1, t2, notes):
new_f = eol_file(f_name, t1, t2, notes)
self.files.append(new_f)

def string(self, indent):
ind = ' '*indent
out = [ind+"%s: %s    %s    %s"%(self.num, str(self.v_pub), str(self.v_crt), str(self.v_mod))]
for i in self.files:
out.append(i.string(indent+4))
return "\n".join(out)

def __repr__(self):
out = ["%s: %s    %s    %s"%(self.num, str(self.v_pub), str(self.v_crt), str(self.v_mod))]
for i in self.files:
out.append("    "+i.__repr__())
return "\n".join(out)

class eol_file:
def __init__(self, f_name, t1, t2, notes):
self.name = f_name
self.f_create = datetime.strptime(t1, "%Y-%m-%d %H:%M:%S")
self.f_revise = datetime.strptime(t2, "%Y-%m-%d %H:%M:%S")
self.f_notes  = notes

def string(self, indent):
out = ' '*indent
return out+self.name

def __repr__(self):
return self.name

def import_eol(fname):
f = open(fname)
f_reader = csv.reader(f, delimiter='|')
f_reader.next()
header = [i.strip() for i in f_reader.next()[1:-1]]
f_reader.next()

datasets = {}
for i in f_reader:
row = [j.strip() for j in i[1:-1]]
if len(row) < 10:
continue
if int(row[0]) not in datasets:
datasets[int(row[0])] = dataset(int(row[0]), row[1])

datasets[int(row[0])].add_file(row[2], row[3], row[4], row[5], row[6], row[7], row[8], row[9])
return datasets

if __name__ == "__main__":
filedir  = "/data/EOL/"
#filename = "dataset_files_version_metadata.txt"
filename = "EOL_dataset_version_metadata.txt"

datasets = import_eol(filedir+filename)
print len(datasets.keys())

num_ver = {}
for i in datasets.keys():
how_many_versions = len(datasets[i].versions.keys())
if how_many_versions not in num_ver:
num_ver[how_many_versions] = 1
else:
num_ver[how_many_versions] += 1
for i in sorted(num_ver.keys()):
print i, num_ver[i]
\end{minted}