%%%%%%%%%%%%%%%%%%%%%%%%%%%%%%%%%%%%%%%%%%%%%%%%%%%%%%%%%%%%%%%%%%%
%                                                                 %
%                            CHAPTER THREE                        %
%                                                                 %
%%%%%%%%%%%%%%%%%%%%%%%%%%%%%%%%%%%%%%%%%%%%%%%%%%%%%%%%%%%%%%%%%%%

\chapter{CONCEPTUAL MODEL}

The conceptual model used within this thesis is built around the expression of three core versioning operations: addition, invalidation, and modification.  These three activities can be represented by interacting with three types of concepts: versions, attributes, and changes.  Versions represent the data entities being compared.  These could be two different editions of a book or versions of software.  It is important to understand that a version is an abstraction as it can be represented by multiple physical files.  In the sections that follow, operations will only consider the interaction between two versions and will be explained later in the chapter.  Versions then contain attributes representing a quantity being modified.  Specifically for tabular data, attributes would correspond to an identifier that refers to particular rows or columns within the data.  Attributes of the two versions are then connected by a change.  This link functions as a very general concept which can be subclassed into more informative types such as unit changes, improving the expressivity of the model beyond PROV's revisionOf concept.

\section{ADDITION}

A difficulty with comparing provenance graphs is that two data objects can have identical structures, but be added to the dataset 

\section{INVALIDATION}

\section{MODIFICATION}
