%%%%%%%%%%%%%%%%%%%%%%%%%%%%%%%%%%%%%%%%%%%%%%%%%%%%%%%%%%%%%%%%%%%
%                                                                 %
%                            CHAPTER THREE                        %
%                                                                 %
%%%%%%%%%%%%%%%%%%%%%%%%%%%%%%%%%%%%%%%%%%%%%%%%%%%%%%%%%%%%%%%%%%%

\chapter{MODEL SPECIFICATION}\label{ch:model}

\section{Introduction}

A versioning data model needs to address a variety of needs not met by provenance models.
In PROV-O and PAV, the modeled entities are exclusively one-dimensional with each version leading sequentially to the next one.
The HCLS model, Figure \ref{HCLSModel}, and Barkstrom model, Figure \ref{hierarchy}, however, display a more complex two-dimensional hierarchy.
The tree models better capture the tiered granularity separating different versions which can result from a higher-tier macro change.
These models also tightly couple new objects with changes to their underlying attributes.
The tiered approach more clearly explains the scale on which two objects within the tree differ.

Provenance models provide concepts to sequentially order data objects but lack the ability to convey differences between farther spanning objects.
In Figure \ref{hierarchy}, the left-most leaf node and the right-most leaf node differ by three changes at the data product level.
A provenance model would need to rely on qualified properties to connect further annotations and describe the higher level changes.
Remember that a common function of versioning systems is to provide a method to determine the amount of change or difference between two objects of a work.
Much of the differences become lost when compressed into a single relation in a provenance graph.
Additional annotations are often in natural language and do not provide a regular attribute to quantify.

The provenance models, on the other hand, do a much better job in explicitly defining the connection between objects which the tree models imply with structure.
The versioning model must contain a mechanism to convey how changes to parts of an object contribute to that object's transition into a new version.
The fundamental operations---\textbf{add}, \textbf{invalidate}, and \textbf{modify}---are used by the model to capture change in a more detailed manner.
These details provide a mechanism to measure change between versions with better clarity than current methods.

\section{Initial Approaches}

The first approach, seen in Figure \ref{DiscardedFig}, simply extends the provenance relation with additional concepts to capture more types of relationships.
\begin{figure}
	\centering
	\vspace{0.0in} % normally the command here would be \includegraphics
	%	\includegraphics{figures/Addition.png}
	\begin{tikzpicture}[every node/.style={draw, rectangle}]
	\begin{scope}[node distance=10mm and 30mm]
	\node (1) [scale=1.25] at (0,0) {Version 1};
	\node (a) [below=of 1, scale=1.25] {Attribute};
	\node (p1) [below=of a, scale=1.25] {Pre-Value};
	\node (p2) [below right=of a, scale=1.25] {Post-Value};
	\node (n) [above=of p2, scale=1.25] {New};
	\node (o) [right=of n, scale=1.25] {Old};
	\node (2) [above =of o, scale=1.25] {Version 2};
	
	\draw [line width=2pt, ->] (1) -- (2);
	\draw [line width=2pt, ->] (2) -- (n);
	\draw [line width=2pt, ->] (2) -- (o);
	\draw [line width=2pt, ->] (1) -- (a);
	\draw [line width=2pt, ->] (2) -- (a);
	\draw [line width=2pt, ->] (a) -- (p1);
	\draw [line width=2pt, ->] (a) -- (p2);
	\end{scope}
	\end{tikzpicture}
	\caption{Provenance oriented versioning model.}
	\label{DiscardedFig}  % the \label command comes AFTER the caption
\end{figure}
Until the introduction of or comparison with Version 2, none of the concepts in Version 1 can be considered new or old.
As the responsible party for introducing changes, Version 2 becomes associated with New, Old, and modified attributes.
Version 1 also relates to modified attributes since it provides the pre-value used to contextualize Version 2's post-value.
The pre and post values are included so that a user can see how much the attribute has changed, much like with a change log.

Adding the attributes as concepts to the model addresses PROV's and PAV's flat approach to version relations, but the attributes do not capture the inter-relation between objects for New and Old attributes.
Having Version 2 be responsible for all the changes causes issues with the model since it must be associated with attributes from an entirely different object.
The Old attributes should not exists within Version 2.
Associating Old attributes with Version 1 would be more appropriate and intuitive to understand.
The model does not capture the type of change, making the result a listing of attributes without the version differences to contextualize the relationship between the versions.

From a different direction, Figure \ref{DiscardedFig2} shows a model created by starting from the change log.
\begin{figure}
	\centering
	\vspace{0.0in} % normally the command here would be \includegraphics
	%	\includegraphics{figures/Addition.png}
	\begin{tikzpicture}[every node/.style={draw, rectangle}]
	\begin{scope}[node distance=10mm and 20mm]
	\node (l) [scale=1.25] at (1,0) {Log};
	\node (n) [above right=of l, scale=1.25] {New};
	\node (o) [below right=of l, scale=1.25] {Old};	
	\node (a) [right=of l, scale=1.25] {Attribute};
	\node (c) [right=of a, scale=1.25] {Change};
	\node (t) [right=of c, scale=1.25] {Type};
	\node (p1) [above right=of c, scale=1.25] {Pre-Value};
	\node (p2) [below right=of c, scale=1.25] {Post-Value};
	
	\draw [line width=2pt,->] (l) -- (n);
	\draw [line width=2pt,->] (l) -- (o);
	\draw [line width=2pt,->] (l) -- (a);
	\draw [line width=2pt,->] (a) -- (c);
	\draw [line width=2pt, ->] (c) -- (t);
	\draw [line width=2pt, ->] (c) -- (p1);
	\draw [line width=2pt, ->] (c) -- (p2);
	\end{scope}
	\end{tikzpicture}
	\caption{Change log based versioning model.}
	\label{DiscardedFig2}  % the \label command comes AFTER the caption
\end{figure}
Attributes are attached to the log as the primary indicators for old, new, and modified concepts.
Change logs often break down and group changes by attributes.
For modified attributes, an additional change concept is associated, encapsulating the values and nature of the change.
At the far right side of the figure is a concept called Type which indicates more specifically the nature of the change for example a unit of speed to another.
Pre and post values are also included to explicitly define the change concept.

The primary drawback of the log-based construction is that the change log assumes all responsibility for every modification to the data even though the document only reports the differences.
The version objects are also left out of the model, leaving the log concept in possession of the attributes.
One of the major breakthroughs with this model construction is that while specific values are kept in the log, those values do not need to be in the model.
By encoding the type of change, the need for actual values becomes superfluous as change type is more generalizable across domains and contexts.

Figure \ref{DiscardedFig3} combines the provenance and change log approaches by capturing the transition from Version 1 to Version 2 in the change log.
\begin{figure}[b]
	\centering
	\vspace{0.0in} % normally the command here would be \includegraphics
	%	\includegraphics{figures/Addition.png}
	\begin{tikzpicture}[every node/.style={draw, rectangle}]
	\begin{scope}[node distance=10mm and 30mm]
	\node (l) [scale=1.25] at (0,0) {Log};
	\node (1) [left=of l, scale=1.25] {Version 1};
	\node (2) [right=of l, scale=1.25] {Version 2};
	\node (n) [below=of 1, scale=1.25] {New};
	\node (o) [below=of 2, scale=1.25] {Old};	
	\node (m) [below=of l, scale=1.25] {Modified};
	\node (a) [below=of m, scale=1.25] {Attribute};
	
	\draw [line width=2pt,->] (1) -- (l);
	\draw [line width=2pt,->] (l) -- (2);
	\draw [line width=2pt,->] (l) -- (m);
	\draw [line width=2pt,->] (l) -- (n);
	\draw [line width=2pt, ->] (l) -- (o);
	\draw [line width=2pt, ->] (m) -- (a);
	\end{scope}
	\end{tikzpicture}
	\caption{Hybrid provenance and change log versioning model.}
	\label{DiscardedFig3}  % the \label command comes AFTER the caption
\end{figure}
The idea here is to enable distance capture between versions by encapsulating all changes within the log concept.
The changes are then associated with specific attributes.
Pre and post values do not appear in the model as knowing a change has occurred and what kind is more valuable than knowing the explicit values involved.
As a data set becomes more volatile, more values would need to be stored, resulting in more of a copy of the data rather than a summarization of the changes.
Notice in Figure \ref{DiscardedFig3} that Attribute has now become disconnected from either version.
Reconnecting the Attribute concept brings into question which version it should be associated with since it exists in both.
The larger issue with both the log based and hybrid approach is that the model resembles a tree more than a graph, making linked data queries less powerful as most of the concepts are disconnected.

A fourth formulation, in Figure \ref{DiscardedFig4}, leverages the insight that when a change interacts with an attribute, the attribute is different in the next version.
\begin{figure}
	\centering
	\vspace{0.0in} % normally the command here would be \includegraphics
	%	\includegraphics{figures/Addition.png}
	\begin{tikzpicture}[every node/.style={draw, rectangle}]
	\begin{scope}[node distance=20mm and 20mm]
	\node (c) [scale=1.25] at (1,0) {Change};
	\node (1) [above left=of c, scale=1.25] {Version 1};
	\node (2) [above right=of c, scale=1.25] {Version 2};
	\node (a1) [below =of 1, scale=1.25] {Attribute 1};
	\node (a2) [below =of 2, scale=1.25] {Attribute 2};
	
	\draw [line width=2pt,->] (a1) -- (c);
	\draw [line width=2pt,->] (a2) -- (c);
	\draw [line width=2pt, ->] (1) -- (a1);
	\draw [line width=2pt, ->] (2) -- (a2);
	\draw [line width=2pt,->] (c) -- (1);
	\draw [line width=2pt,->] (c) -- (2);
	\end{scope}
	\end{tikzpicture}
	\caption{Highly connected model of just versions, changes, and attributes}
	\label{DiscardedFig4}  % the \label command comes AFTER the caption
\end{figure}
The model addresses the attribution problem by forming two attributes, each associated with a different version.
These attributes inform a change which acts upon both version concepts.
The Log object is dropped for the model since it is a method to convey change and not an actor involved in the change.
From the highly connected construction, new and old attributes no longer need to be explicitly stated, but they can be implied from the model's structure.
A new attribute would not exist in Version 1 so Attribute 1 and its associated properties (arrows) are removed, leaving a unique construction implying an attribute addition.

One observation is that the relation from changes to versions is redundant since the links from version to attribute to change implies the same relationship.
Removing the explicit relation would shorten the number of triples required to encode a change and improve scalability.
The versioning graph using this highly connected model would also be easier to query if the edges were oriented in the same direction, additionally implying that change flows from one version to the next.
These final observations result in the current versioning model.

\section{Model Objects}

The versioning model incorporates three kinds of objects: \textbf{versions}, \textbf{attributes}, and \textbf{changes}.
A \textbf{version} object represents the items being compared such as a book or spreadsheet.
In PROV, a \textbf{version} would likely correspond with the \textit{prov:Entity} involved in a \textit{prov:wasRevisionOf} property.
The \textbf{attribute} object refers to specific parts which make up a \textbf{version}.
\textbf{Attributes} could be lines in a book or columns in a spreadsheet.
Including \textbf{attributes} addresses the lack of detail involved in a \textit{prov:wasRevisionOf} or \textit{pav:previousVersion}.
The relationship between \textbf{versions} and \textbf{attributes} captures the influence that changes in the underlying part will have on the overarching \textbf{version}.
Because the model refers to specific parts of a \textbf{version}, the \textbf{version} concept corresponds most closely with a FRBR \textbf{manifestation} rather than an \textbf{expression}.
The presence or absence of an \textbf{attribute} is used to determine the kind of \textbf{change} which occurs to the \textbf{attribute} between \textbf{versions}.
\textbf{Changes} are used to link together \textbf{attributes} from different \textbf{versions}.
The \textbf{change} captures a difference between the old \textbf{version} state and the new \textbf{version} state.
While the \textbf{change} object greatly resembles a PROV qualified property, its form can change depending on the kind of \textbf{change}, like a \textit{schema:UpdateAction}.

\subsection{Left-hand Right-hand Convention}

In the following diagrams and figures, the original or base version and its attributes will be placed on the left-hand side and the new version will be placed on the right-hand side with its attributes.
References to the versions as previous and next are avoided since sequencing may not play a major role in distinguishing versions.
Scientific data in large repositories often track sequential releases of data, but a book may have different versions distinguished by printed language.
To recognize this distinction, objects will be referred to as the left-hand \textbf{version} or left-hand \textbf{attribute} when they are not sequentially or temporally related.

\section{Model Changes}

The model bases \textbf{changes} around the three core versioning operations because their commonality across systems provides a fundamental basis for comparisons.
\textbf{Additions} occur when an \textbf{attribute} appears only in the right-hand \textbf{version}.
When an \textbf{attribute} only shows up in the left-hand \textbf{version}, the model captures this as an \textbf{invalidation}.
Finally, a \textbf{modification} change has \textbf{attributes} in both the left and right-hand \textbf{versions}, but it only connects two \textbf{attributes} if their values are different.
These three combinations cover the possible situations within the model.

\begin{figure}
	\centering
	\vspace{0.0in} % normally the command here would be \includegraphics
	%	\includegraphics{figures/Addition.png}
	\begin{tikzpicture}[every node/.style={draw, rectangle}]
	\begin{scope}[node distance=20mm and 20mm]
	\node (c) [scale=1.25] at (1,0) {Change M};
	\node (1) [above left=of c, scale=1.25] {Version 1};
	\node (2) [above right=of c, scale=1.25] {Version 2};
	\node (a1) [below =of 1, scale=1.25] {Attribute 1};
	\node (a2) [below =of 2, scale=1.25] {Attribute 2};
	
	\draw [line width=2pt,->] (a1) -- (c);
	\draw [line width=2pt,->] (c) -- (a2);
	\draw [line width=2pt, ->] (1) -- (a1);
	\draw [line width=2pt, ->] (2) -- (a2);
	\end{scope}
	\end{tikzpicture}
	\caption{Model of the relationships between Versions 1 and 2 when modifying Attribute 1 from Version 1 as a result of Change M, resulting in Attribute 2 from Version 2}
	\label{ModificationFig}  % the \label command comes AFTER the caption
\end{figure}

\subsection{Modification}

The \textbf{modification} relation occurs when an \textbf{attribute} appears in both \textbf{versions} and their values are different.
In Figure \ref{ModificationFig}, a \textbf{modification} is captured between two versions.
Each \textbf{version} has an \textbf{attribute}, Attribute 1 and Attribute 2, respectively.
Finally, a \textbf{change} object connects the two \textbf{attributes}, denoting that the values described by the attribute are different.

The specific values pertaining to Attribute 1 and Attribute 2 are not captured by the model because acknowledging that a difference exists is more important.
Extending the model to properly communicate the significance of a modification for a wide variety of domains would require sizable domain knowledge and would be outside the scope for this project.
In addition, the model would essentially begin storing a copy of the data set, leading to space and redundancy concerns.

\subsection{Addition}

In Figure \ref{AdditionFig}, the \textbf{addition} model differs from the \textbf{modification} construction by the absence of Attribute 1.
\begin{figure}
	\centering
	\vspace{0.0in} % normally the command here would be \includegraphics
	%	\includegraphics{figures/Addition.png}
	\begin{tikzpicture}[every node/.style={draw, rectangle}]
	\begin{scope}[node distance=20mm and 20mm]
	\node (c) [scale=1.25] at (1,0) {Change A};
	\node (1) [above left=of c, scale=1.25] {Version 1};
	\node (2) [above right=of c, scale=1.25] {Version 2};
	\node (a) [below =of 2, scale=1.25] {Attribute 2};
	
	\draw [line width=2pt,->] (1) -- (c);
	\draw [line width=2pt,->] (c) -- (a);
	\draw [line width=2pt, ->] (2) -- (a);
	\end{scope}
	\end{tikzpicture}
	\caption{Model of the relationships between Versions 1 and 2 when adding an Attribute 2 to Version 2 as a result of Change A}
	\label{AdditionFig}  % the \label command comes AFTER the caption
\end{figure}
The absence creates a disconnect between ``Version 1" and ``Change A".
We know that a connected graph will be desirable to accommodate traversal using linked data query languages so ``Version 1" must be reconnected to the other concepts in the model.
A property is used to create a path between the two \textbf{attributes} to indicate the contribution of  ``Version 1" to the change's lineage.
The path does not show that ``Version 1" informs or creates ``Attribute 2", while that may be true.
The construction was also chosen to create a symmetric orientation with the \textbf{invalidation} change.


\subsection{Invalidation}

The \textit{invalidation} model has a missing \textbf{attribute} on the right-hand side of the relation, contrary to the \textbf{addition} construction.
As a result of the invalidation, an attribute no longer exists in the right-hand \textbf{version}.
As seen in Figure \ref{InvalidationFig}, the invalidation change concept matches to the Version 2 object.
\begin{figure}[b]
	\centering
	\vspace{0.0in} % normally the command here would be \includegraphics
	%	\includegraphics{figures/Addition.png}
	\begin{tikzpicture}[every node/.style={draw, rectangle}]
	\begin{scope}[node distance=15mm and 20mm]
	\node (c) [scale=1.25] at (1,0) {Change I};
	\node (1) [above left=of c, scale=1.25] {Version 1};
	\node (2) [above right=of c, scale=1.25] {Version 2};
	\node (a) [below =of 1, scale=1.25] {Attribute 1};
	
	\draw [line width=2pt,->] (a) -- (c);
	\draw [line width=2pt,->] (c) -- (2);
	\draw [line width=2pt, ->] (1) -- (a);
	\end{scope}
	\end{tikzpicture}
	\caption{Model of the relationships between Versions 1 and 2 when invalidating Attribute 1 from Version 1 as a result of Change I}
	\label{InvalidationFig}  % the \label command comes AFTER the caption
\end{figure}
Just like in \textbf{addition} model, this construction maintains a link between the two \textbf{version} objects.
In this case, it makes more conceptual sense, however, because ``Version 2" invalidates ``Attribute 1" by omitting it.


\section{Summary}

The versioning model provides a method to capture change information in greater detail than current provenance models.
The inclusion of \textbf{versions} and \textbf{attributes} into the model connect changing items with the objects they influence.
The \textbf{changes} create a ladder-like structure to connect together \textbf{version} objects in greater detail.
Each rung of the ladder can not only be counted, but also grouped into types of change according to the respective operation.
The method of instantiating a versioning graph will be covered in Chapter \ref{ch:implement}.