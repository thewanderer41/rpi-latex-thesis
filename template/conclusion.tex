\chapter{CONCLUSION}

Formalizing versioning into a linked data model with a state-based approach and relating version change with changes in their attributes will improve the ability to standardly communicate version information.
The proposed model does not possess the same brevity found in most popular provenance models, but it captures versioning as groups of relationships rather than sets of activities.
This reveals a relationship between versions and their attributes in informing version changes.
The approach led to difficulties viewing encoded versioning information in change logs due to excessive file sizes, but analysis suggests that there may be methods to merge multiple entries and improve performance.
Future groups will need to perform further work and expand data change log adoptions and computability.
The model also demonstrated that data distributors must find specific quantifiable measures to inform version identifiers since they play a key role in communicating information change.
Direct edit counts, however, do not present a nuanced view of change impacts and should be viewed as initial characterizations of a version.
Finally, we have demonstrated that versioning information can be used to perform services beyond documentation and tracking once a standard formal model has been produced.
While often overlooked, version capture formalization will play a major role in data set evolution.